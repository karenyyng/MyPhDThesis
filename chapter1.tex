% LaTeX template file for a UC Davis Mathematics / Physics Ph.D. Dissertation
% The UC Davis Dissertation Formatting Requirements can be found at
% http://www.gradstudies.ucdavis.edu/students/filing.html

\documentclass[ucdthesis.tex]{subfiles}
    
\begin{document}
    % The following commands produce page numbering at the bottom
    % center using roman numerals per UC Davis requirements for the
    % front matter of the dissertation:
    % \pagenumbering{roman}
    % \pagestyle{plain}
    % The following command produces double-spaced lines for the
    % remainder of the document:
    \doublespacing

		\chapter{Motivation for studying Dark Matter (DM) at different 
			scales}		
		\label{chapter:1}
    \epigraph{``If your experiments require statistics, you ought to have done a
		better experiment.''}{Ernest Rutherford} 

		\section{Research background}
		% Argues why studying DM is important
		\subsection{Existing DM models, the successes and pitfalls}
		Dark matter compose of $\sim$80\% of the matter density in our universe
		(CITE). 
		Interacts gravitationally.	
		Its gravity pull together baryonic matter and counteracted 
		Dark Energy.  

		% Explain the most successful model of DM, and the successes of CDM 
		% what properties of DM do we know now?
		The Cold Dark Matter model (CDM) matches observations the best at large scale

		% The pitfalls of CDM 
		% what properties of DM do we still not know?
		Why are other DM models, e.g. warm dark matter, ruled out?

		At smaller scales, there are multiple low signal-to-noise observations that 
		are in contention with CDM. Missing satellite problem, cusp-and-core problem 
		offset between DM and the galaxy counterparts in cluster mergers.

		% Explain that only astrophy observations can probe self-interaction of DM
		% particles experiments are not successful and they probe
		% only the baryon-DM cross section,  
		Possible candidates have small interaction cross-sections. 
		Ground  experiments have inconclusive evidence due to the low
		signal-to-noise (SN).

		\subsection{Techniques of detecting DM distribution - weak lensing}
		Astrophysical observations also have their limitations and a separate sets
		of systematics and unknowns.

		\section{Research studies and their significance}
		\subsection{Why use probabilistic methods?}
		No other way to do a better experiment, but we have a choice to use models
		that can represent the uncertainties sufficiently.
		One uniting theme of the thesis are the choice of probabilistic methods
		among the several studies. Monte Carlo sampling, Markov Chain Monte Carlo, 
		mixture models, kernel density estimations, graphical models and 
		Gaussian Processes. 

		What are probabilistic methods?
		How to improve existing methods that utilizes estimators. 
		Discards information 

		How to best incorporate prior information during the modeling steps. 
		Methods showing explicit modeling choices for comparison. 
		
		\begin{itemize}
		\item model selection / comparison problem
		\item whether the model can be generalized, or in terminology of Machine
			Learning, whether we overfit our model to the data 
		\item hierarchical models 
		\item uncertainty propagation 
		\item make assumptions explicit for future improvements, a recipe for writing
		inference algorithms - account for other possible causes of signal 
		- what needs to be jointly modeled 
		\end{itemize}

		% Primary "direct" observation of DM is from the lensing signal 
		% Explain the basics of weak lensing 
		\subsection{Monte Carlo simulation of a merger of galaxy cluster}
		Introduce Weak lensing as an observable for density and mass estimation of
		clusters. In Chapter 2, mass-estimation by fitting parametric NFW models. 
		Gives an overview of the relationship of DM with respect to the other
		components of a galaxy cluster during a merger. Combine weak lensing with
		other observables in a Monte Carlo sampling scheme.

		\subsection{Examining how DM distribute differently from galaxies under no
		DM self-interaction}
		In Chapter 3, I examine the distribution of Dark Matter in galaxy clusters
		from a entirely new setting, using cosmological data in the Illustris
		simulation. Show the deficiency of statistical estimators 

		\subsection{Modeling large-scale DM distribution with Gaussian Process}
		At large scales of the universe, weak lensing also serves as a probe of 
		cosmological parameters. Constrains $\Omega_m$ and $\sigma_8$ 
		One of the primary goals of the next generation sky survey, the Large
		Synoptic Sky Survey (LSST). In Chapter 4, I will explain a new generative
		model for extracting information from cosmic shear. 
	  Has less restrictions on prior assumptions for cosmological models.	


        
	  \bibliography{chapter1}{}

\end{document}
