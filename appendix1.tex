% LaTeX template file for a UC Davis Mathematics / Physics Ph.D. Dissertation
% The UC Davis Dissertation Formatting Requirements can be found at
% http://www.gradstudies.ucdavis.edu/students/filing.html

\documentclass[ucdthesis.tex]{subfiles}
\renewcommand{\chaptername}{Appendix}
\usepackage{xr}
\externaldocument{chapter2}
    
\begin{document}
    
    % The following commands produce page numbering at the bottom
    % center using roman numerals per UC Davis requirements for the
    % front matter of the dissertation:
    \pagenumbering{roman}
    \pagestyle{plain}
    % The following command produces double-spaced lines for the
    % remainder of the document:
    \doublespacing
		\chapter{Supplementary material of the study of El Gordo}{}{}
		\label{appendix1}

						 
		\cfoot{\thepage}
		\pagenumbering{arabic}

    \section{Default weights used for dawson's monte carlo simulation}
    \label{app:priors}
    The default weight that we employed can be summarized as
    follows for most of the output variables: 
    \begin{equation}
    	w(v_{3D}(t_{\rm per})) = 0\text{ if }v_{3D}(t_{\rm per}) >
    	v_{\text{free fall}}. 
    \end{equation}
    \begin{equation}
    	w(TSP_{\rm out}) = 
    	\begin{cases}
    		& \text{const}~\text{if }TSP_{\rm out} < \text{age of universe at } z=0.87	\\
    		& 0~\text{otherwise}.
    	\end{cases}
    \end{equation}
    In addition, we apply the following weight on $TSP_{\rm ret}$:
    \begin{equation}
    	w(TSP_{\rm ret}) = 
    	\begin{cases}
    		& \text{const}~\text{if }TSP_{\rm ret} < \text{age of universe at } z=0.87	\\
    		& 0~\text{otherwise} \label{eqn:TSM_1}.
    	\end{cases}
    \end{equation}
    To correct for observational limitations, we further convolved the
    posterior probabilities of the different realizations with 
    \begin{equation}
    	w(TSP_{\rm out} | T) = 2 \frac{TSP_{\rm out}}{T},
    \end{equation}
    to account for how the subclusters move faster at lower $TSP$ and thus it
    is more probable to observe the subclusters at a stage with a larger $TSP$.
    \par 
    \section{Plots of outputs of the monte carlo simulation}
    We present the PDFs of the inputs of the dynamical simulation and the
    marginalized PDFs of the outputs after applying the polarization weight in
    addition to the default weights. See Fig.~\ref{fig:plot_config} for explanations of
    the order that we present the figures containing the PDFs . 
    \begin{figure}
    	\begin{center}
    	\includegraphics[width=\linewidth]{ElGordo_plot_config.png}
    	\end{center}
    	\caption{Matrix of variables used in the simulations. We present them in
    	4 categories, including, inputs, geometry, time and velocity. Regions of
    	the same color represent one plot and the number
    indicates the corresponding figure number in this appendix.
    \label{fig:plot_config}
    }
    \end{figure}
    \label{app:results}
    %%%%%%%%%%%%% TASK --- 
    \clearpage
    \begin{figure*}
    	\begin{minipage}{180mm}
    	\begin{center}
    	\includegraphics[width=0.65\linewidth]{TwoMnWBSG_inputsVsinput.png}
    	\caption{Marginalized 2-dimensional PDFs of original inputs (vertical axis) 
    		and the inputs after applying polarization weight and default weights 
    		(horizontal axis). The inner and outer contour
    denote the central 68\% and 95\% confidence regions respectively.
    The circular contours show that the application of weights did not introduce
    uneven sampling of inputs. }
    	\end{center}
    	\end{minipage}
    \end{figure*}
    \begin{figure*}
    \begin{minipage}{180mm}
    	\begin{center}
    	\includegraphics[width=0.5\linewidth]{TwoMnWBSG_tri_geo.png}
    	\caption{One-dimensional marginalized PDFs (panels on the diagonal) and
    		two-dimensional marginalized PDFs of variables
    		denoting characteristic distances and projection angle of the mergers.
    	\label{fig:geom_geom}
    	}
    	\end{center}
    	\end{minipage}
    \end{figure*}
    \begin{figure*}
    \begin{minipage}{180mm}
    	\begin{center}
    	\includegraphics[width=0.7\linewidth]{TwoMnWBSG_geoVSinputs.png}
    	\caption{Marginalized PDFs of characteristic distances and projection
    		angle of the merger and the inputs of the simulation.}
    	\end{center}
    	\end{minipage}
    \end{figure*}
    \begin{figure*}
    \begin{minipage}{180mm}
    	\begin{center}
    	\includegraphics[width=0.5\linewidth]{TwoMnWBSG_tri_time.png}
    	\caption{One-dimensional PDFs of characteristic timescales of the simulation
    (panels on the diagonal) and the marginalized PDFs of different
    timescales. Note that there is a default weight for constraining $TSP_{\rm out}$ but
    not for $TSP_{\rm ret}$ and $T$, so $TSP_{\rm out}$ spans a smaller range.}
    	\end{center}
    \end{minipage}
    \end{figure*}
    \begin{figure*}
    \begin{minipage}{180mm}
    	\begin{center}
    	\includegraphics[width=0.5\linewidth]{TwoMnWBSG_timeVsgeo.png}
    	\caption{Marginalized PDFs of characteristic timescales of the simulation
    and the characteristic distances and the projection angle of the merger. }
    	\end{center}
    \end{minipage}
    \end{figure*}
    \begin{figure*}
    \begin{minipage}{180mm}
    	\begin{center}
    	\includegraphics[width=0.7\linewidth]{TwoMnWBSG_timeVSinput.png}
    	\caption{Marginalized PDFs of characteristic timescales of the simulation
    and the inputs.}
    	\end{center}
    \end{minipage}
    \end{figure*}
    \begin{figure*}
    \begin{minipage}{180mm}
    	\begin{center}
    	\includegraphics[width=0.5\linewidth]{TwoMnWBSG_tri_vel.png}
    	\caption{One-dimensional marginalized PDFs of velocities at
    	characteristic times (panels on the diagonal) and marginalized PDFs of
    different velocities.}
    	\end{center}
    \end{minipage}
    \end{figure*}
    \begin{figure*}
    \begin{minipage}{180mm}
    	\begin{center}
    	\includegraphics[width=0.5\linewidth]{TwoMnWBSG_velVStime.png}
    	\caption{Marginalized PDFs velocities and the characteristic timescales
    	of the simulation against the inputs.}
    	\end{center}
    \end{minipage}
    \end{figure*}
    \begin{figure*}
    \begin{minipage}{180mm}
    	\begin{center}
    	\includegraphics[width=0.5\linewidth]{TwoMnWBSG_velVSgeo.png}
    	\caption{Marginalized PDFs of the velocities at characteristic timescales
    		and the characteristic distances and the projection angle of the merger. }
    	\end{center}
    \end{minipage}
    \end{figure*}
    \begin{figure*}
    \begin{minipage}{180mm}
    	\begin{center}
    	\includegraphics[width=0.7\linewidth]{TwoMnWBSG_velVSinputs.png}
    	\caption{Marginalized PDFs of relative velocities characteristic
    	timescales of the simulation and the inputs.}
    	\end{center}
    \end{minipage}
    \end{figure*}
    
    \section{Comparison of the outgoing and returning scenario}
    \label{app:Bayes_factor}
    Here, we compare the different merger scenarios using the two relics
    independently and show that they consistently give the conclusion that the returning
    scenario is favored for the plausible range of $\beta$. For each merger
    scenario, we compute
    the (marginalized) probability of producing simulated
    values ($s_{proj}$) compatible with the observed location of the radio
    relic ($s_{obs}$). 
    
    Quantitatively, we want to compute and compare the probability:  
    \begin{align} 
    	&P(s_{proj} \text{ compatible with }s_{obs} | M)  \label{eqn:prob}\\
    	&=\iint f(S_{proj} \cap S_{obs} | M, \beta) f(\beta | M) d s_{proj} d\beta\\
    	&=\iint  f(s_{proj}|M, \beta) f(s_{obs}) f(\beta | M) d s_{proj}
    	d\beta, 
    \end{align}
    where $f$ indicates the corresponding PDF, $M$
    represents one of the merger scenarios, and $\beta$ is defined in equation
    ~\ref{eqn:NW_speed}, $s_{proj} \in S_{proj}$ and $s_{obs} \in S_{obs}$. 
    We set our priors set to be uniform for the marginalization: 
    \begin{equation}
    	f(\beta | M_{ret}) = f(\beta | M_{out}) =  
    	\begin{cases}
    		& \text{const}~\text{if } 0.7 \leq \beta \leq 1.5 \\
    		& 0~\text{otherwise}.
    	\end{cases}
    \end{equation}
    which is more conservative than the most likely range of $\beta$, which is
    $0.7 < \beta < 1.1$.
    We found: 
    \begin{align}
    	&P(S_{proj} \cap S_{obs} | M_{ret}) / P(S_{proj} \cap S_{obs} |
    	M_{out})\\
    	&=
     \begin{cases}
      2.1 \text{ for the NW relic},\\
      460 \text{ for the SE relic},
     \end{cases}
    \end{align}
    which shows that the returning scenario is favored over the outgoing
    scenario. 
    
    This test quantity differs from the traditional hypothesis testing or model comparison in several ways: 
    \begin{enumerate}
    \item we did not compute a likelihood function.  We have adopted
    non-parametric PDFs in our Monte Carlo simulation,i.e. there is no well-known
    functional form of the likelihood in our context. We make use of
    $f(S_{proj} \cap S_{obs})$ to penalize the simulated values being different from our observed data
    \item with this quantity, we are not asking whether the expected value of
    	the radio relic such as the mean or median from each model match the
    	observation best. Those estimators take into account the values that do not match the observed location of the radio relic. 
    \item we marginalized the uncertainty in $\beta$ to be as conservative
    	as possible, instead of assuming a fixed value of $\beta$.
    \end{enumerate} \par
    %The second quantity that we compute is the Wald statistic at a range of possible $\beta$ values.
    %Wald statistic is usually used hypothesis testing.
    %Computing the Wald statisic allows us to ask, given a particular model (merger
    %scenario),
    %whether the null value (sample mean) is in the confidence
    %interval \citep{Wasserman04}. The Wald statistic that we compute for each model is: 
    %\begin{align}
    %	W(\beta) = \frac{\bar{s}_{obs} - \mu(\beta)}{\sigma / (n)^{1/2}} 
    %\end{align}
    %where $\bar{x}$ is the sample mean, $\mu(\beta)$, $\sigma$ and $n$ are the
    %population mean, population standard deviation and size of samples of the
    %simulated model respectively. A higher Wald statistic value would represent
    %a larger difference between the observed and model value, while taking into account the model uncertainty. Within the range of
    %most likely $0.7 < \beta < 1.5$, the Wald statistic shows that the observed
    %relic location is more compatible with the confidence interval of the returning
    %scenario. (See Figure~\ref{fig:waldtest})
    
    \begin{figure}
    	\includegraphics[width=\linewidth]{prob_ratio.png}
    	\caption{Probability ratio between the returning model (numerator) and
    		the outgoing model at given $\beta$. We remind readers $\beta$ is a factor
    		relating the {\it time-averaged} shock velocity and the pericenter
    		velocity of the corresponding subcluster.  
    	\label{fig:prob_ratio}}
    \end{figure}
    
    \begin{figure}
			\begin{center}
			\includegraphics[width=0.8\linewidth]{polar_prior_bounds_E.pdf}
				\caption{Comparison of the PDFs of the observed position of the SE relic (red bar
					includes 95\% confidence interval of location of relic in the center of
					mass frame) with the predicted position from the two simulated merger scenarios (blue for outgoing and green for the returning scenario). 
				For the plausible values of $\beta < 1.1$, the returning scenario is preferred. 
				We obtained similar conclusion about the merger scenario as for the NW
				relic calculation.
				\label{fig:positionprior_SE}}
		\end{center}
    \end{figure}
    
    \begin{figure}
			\begin{center}
    	\includegraphics[width=.8\linewidth]{polar_prior_bounds.pdf}
    	\caption{Comparison of the PDFs of the observed position of the NW relic (red bar
    		includes 95\% confidence interval of location of relic in the center of
    		mass frame) with the	predicted position from the two simulated merger scenarios (blue for outgoing and green for the returning scenario). 
    	For the plausible values of $\beta  < 1.1$, the returning model is
    	preferred. For comparison purpose, we also show that $\beta > 1.3$ (top
    	panel) for the outgoing scenario to be favored.  
    	Note that we made use of the polarization weight for producing this figure. 
    	\label{fig:positionprior}}
		\end{center}
    \end{figure}
    \clearpage
    \section{Acknowledgements}
    We thank Franco Vazza, Marcus Br\"{u}ggen and Surajit Paul for sharing
    their knowledge on the simulated properties of radio relic and merger
    shocks. We extend our gratitude to Reinout Van Weeren for first proposing the use of
    radio relic to weight the Monte Carlo realizations. We appreciate the
    comments from Maru\v{s}a Brada\v{c} about using the position of the relic to
    break degeneracy of the merger scenario. KN is grateful to Paul Baines and
    Tom Loredo for discussion of the use of prior information and sensitivity tests. 
    Part of this work was performed under the auspices of the U.S. DOE by LLNL
    under Contract DE-AC52-07NA27344. 
    JPH gratefully acknowledges support from Chandra (grant number GO2-13156X)
    and Hubble (grant number HST-GO-12755.01-A).
    We would also like to thank 
    GitHub for providing free repository for version control for our data and
    analyses. This research made use of APLpy, an open-source plotting package for Python
    hosted at http://aplpy.github.com; Astropy, a community-developed core
    Python package for Astronomy \citep{astropy}; AstroML, a
    machine learning library for astrophysics \citep{VanderPlas2012}, and IPython, a system for
    interactive scientific computing, computing in science and engineering
    \citep{Perez2007}.\par
    Note: The authors have made the Python code for most of the analyses openly
    available at
    \href{https://github.com/karenyyng/ElGordo\_paper1}{https://github.com/karenyyng/ElGordo\_paper1}. 
		
		\bibliography{appendix1}
		% \nobibliography{appendix1}

      
\end{document}
