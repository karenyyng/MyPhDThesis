\newchapter{Statistical modeling, tests and requirements for using galaxy-dark
matter offset to constrain SIDM}{Statistical modeling, tests and requirements
for using galaxy-dark matter offset to constrain SIDM}{}
\label{chapter2}

Like all other experiments that aim to probe the particle properties of
DM, this proposed study that uses galaxy-DM offset as a probe of SIDM suffers from a low signal-to-noise ratio. 
In order to confidently confirm or rule out the effects of SIDM, we need careful
statistical modeling and tests. 
In this chapter, I lay out the bare-bone probabilistic graphical model (model or PGM hereafter) for
capturing the dependence structure of various relevant variables based on the
physics of galaxy cluster mergers.  

I examine the cosmological constraints and the computational requirements.
Then I examine the required number of simulations and observational resolution for making
the model competitive for detecting effects of SIDM.\@ 
\begin{figure}
	\center
	\includegraphics[width=.8\textwidth]{/Users/karenyng/Documents/pgm_MCC/infering_sigma_SIDM_simulation.png}
	\caption{Probabilistic graphical model to denote the dependencies of
		different variables. The dot denotes fixed variable. Circle with white
		background denotes hidden random variables. Circle with grey background
	denotes observed random variable.} 
\end{figure}

\subsection{Variables of the probabilistic graphical model}
$\vec{\theta}_{\Lambda CDM}$ = fixed cosmological parameters from $\Lambda
CDM$, i.e. 
\begin{itemize}
	\item $H_0 = 70~\kilo\meter~\mega {\rm pc}^{-1}~\second^{-1}$, 
	\item $\Omega_M = 0.3$, 
	\item $\Omega_\Lambda = 0.7$
\end{itemize}  
$\sigma_{SIDM}$ = self-interacting cross section, smaller cross section means dark
matter interacts less \\  
$h_{mer}$ = variables that denote merger history of the cluster, 
e.g 
\begin{itemize}
	\item non-relaxedness 
	\item time-since-pericenter
	\item impact parameter
	\item Active Galactic Nuclei (AGN) feedback
	\item gas sloshing that results in oscillation of the mass distribution 
	\item initial merger velocities etc.
\end{itemize}
since we do not have a good handle of these effects we lump them into one
variable $h_{mer}$ for now and examine them one by one later. 
\\    
$I_{g}$ = modeling assumptions or modeling uncertainty, observation window, 
selection function\\ 
$I_{D}$ = modeling assumptions, observation window, uncertainty due to modeling
methods, selection function\\   
$d_{g}$ = data of observed cluster galaxies  \\ 
$d_{D}$ = data of observed background galaxies within the cluster's field of
view  \\ 
$\psi_{g}$ = mass / luminosity distribution of galaxies in 3D  \\ 
$\psi_{D}$ = mass distribution of DM from lensing in 3D \\ 
$\eta_{g}$ = point estimate of 2D projected center of the distribution of galaxies \\ 
$\eta_{D}$ = point estimate of 2D projected center of the distribution of DM \\ 
$|\Delta \eta|$ = Euclidean distance (scalar) between the projected point estimates\\   
$\alpha$ = projection angle\\ 



\section{} 
