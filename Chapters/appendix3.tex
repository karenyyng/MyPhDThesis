\section{Algorithm of the Shrinking aperture estimates}
\label{app:shrink_apert}
\begin{algorithm}
	\caption{Shrinking aperture algorithm with luminosity weights}
	\KwData{subhalos that satisfy cuts as a galaxy}
	 \hrulefill \\

	 initial aperture centroid = weighted mean galaxy location in each spatial dimension\\
 	distance array = euclidean distances between initial aperture center and each galaxy
	location \\
 	aperture radius = 90th percentile of the weighted distance array\\ 
	\While{ (newCenterDist - oldCenterDist) / oldCenterDist $\geq$ 2e-2}{
 		new data array = old data array within aperture\\
 		newCenter = weighted mean value of new data along each spatial dimension 
	}   \hrulefill
\end{algorithm}
\begin{figure*}
	\begin{center}
	\includegraphics[width=0.7\linewidth]{Mass_abundance_relationship.pdf}
	\caption{Cumulative distribution of clusters above a certain mass threshold
		for different samples.
		Each distribution is normalized to the sample size.
		The samples with more than one dominant peak are labeled as $1.2 < \nu < 2.2$. The 
		The samples with approximately one dominant peak are labeled as $\nu
		\leq 1.2$.  
		If the subsets have the same cluster mass abundance as the full sample,
		the three plots should lie on top of one another.
		\label{fig:mass_abundance_distribution}
	}
\end{center}
\end{figure*}
 

\section{Table of results}
\label{app:table_of_results}
\begin{table}
	\begin{center}
	\ifthenelse{\boolean{thesis}}{
	\caption{Properties of the clusters used in the analysis. Richness is
	computed based on $i-$band $< 24.4$ assuming $z=0.3$.\label{tab:cluster_prop}
		The table is too large to be included inside the dissertation and is instead
	available at \href{https://goo.gl/uGUhec}{https://goo.gl/uGUhec}}
}{
	\caption{Properties of the clusters used in the analysis. Richness is
	computed based on $i-$band $< 24.4$ assuming $z=0.3$.\label{tab:cluster_prop}
}
}
\end{center}
\end{table}



\begin{landscape}
\begin{table*}
	\begin{center}
	\caption{Summary statistic characterizing the offset distributions
		between the most bound particle and various summary statistics of 
		the member galaxy population. There are explanations of the outliers in
		this table in the result subsection \ref{subsec:galaxyDMoffset}.
	\label{tab:most_bound_particle_offset_distributions}}
	\input{Chapters/most_bound_particle_table.tex}
\end{center}
\small{The offsets represented with the prime $'$ symbols are estimated using the luminosity weighted galaxy 
data.}
\end{table*}
\end{landscape}


\begin{table}
	\begin{center}
	\caption{Summary statistic characterizing the offset distributions
		for between the DM peak and the estimated galaxy location. 
		All 43 clusters and all 768 projections are used in this table. 
		The highest density values were used for the computation when there were more
		than one peak value estimated from the KDE. There are different levels of 
		asymmetry depending on how sparse a region is. 
	\label{tab:offset_distributions}}
	\input{Chapters/full_sum_stat_table}
\end{center}
\small{The offsets represented with the prime $'$ symbols are estimated using the luminosity weighted galaxy 
data.}
\end{table}


