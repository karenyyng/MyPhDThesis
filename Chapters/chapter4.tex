
This work was performed as a continuation of \citep{Schneider2014}  
and forms the basis of all the analyses performed in Schneider et al. 2016.

\section{Introduction} 
% what parameter does cosmic shear constrain?
% why there needs to be an alternative of cosmic shear inference - simulation
% for inferring the non-Gaussian component  

Importance of $\sigma_8$ and $\sigma_M$  


Our universe, on the largest scales, contains matter fluctuations ($> 100
h^{-1}$ Mpc) that are
mainly Gaussian. 
% This is consistent with observed low power mode of the
% cosmic microwave background (CMB).  
However, for the late-time evolution of the matter fluctuations on smaller
scales ($< 100 Mpc$), non-linear process such as gravitational collapse introduced  
perturbations that cannot be well described by Gaussian statistics.   
Many studies have stated the necessity to use higher-order statistic to represent the
non-Gaussianity \cite{Jee} 
Sufficient statistic 

Our method provides an alternative from having to use the two-point correlation 
function or the bispectrum function for representing cosmic shear information. 
have to evaluate the non-Gaussian component from N-body simulations  

We have developed an interpolation scheme that allows us to encode the 
lensing physics.  
In this work, we 
1) show the basic steps for incorporating lensing physics in 
a suitable covariance kernel form for a Gaussian Process  
2) lay out the statistical model for performing cosmic shear inference 
with the Gaussian Process with modified kernel 
3)  

\section{Method}

In the weak lensing regime, we assume  


A Gaussian Process is the generalization of the concept of a multivariate Gaussian
to higher dimension \citep{Rasmussen2006}. 

a non-parametric linear smoother \cite{Hastie1990} that is closely   



The two hyperparameters Provide the characteristic length  

% Fig. 1 - Put plot that shows different realizations from a Gaussian Process  


Instead of drawing 

We propose a generative statistical model by incorporating the relevant lensing physics.  
% Simultaneously infer both the shear and the convergence component
% interpolate between 

\subsection{Interpretation of the Gaussian Process hyperparameters}
 

\section{Results}
Our work is the first attempt to represent the information in cosmic shear
without assuming a Gaussian distribution for the density modes.  

We can simultaneously draw the 


\section{Discussion}
The constraints on cosmological parameters need to be verified in a mock
weak-lensing analysis of cosmological simulation. 
It enables the separation of E and B mode in the convergence field  


\section{Conclusion}

