

\section{Introduction} 
During the latest stage of structure formation, the universe gave birth to
non-linear, hierarchical structures known as galaxy clusters. 
These clusters, made up of dark matter, galaxies and hot gas,
are constantly accreting mass, merging and evolving with their
environments. Bright galaxies that belong to a galaxy cluster or group, in 
particular, highlight the overdensities of the underlying dark matter (DM) 
distribution. 

In these dense regions of the clusters, the rates of particle
interactions can be enhanced, including the long-suspected self-interaction of DM
particles (hereafter, SIDM).  
Many papers have used the offsets between the summary statistics of the DM
density and the galaxy density to give constraints on 
the self-interaction cross
section, i.e. $\sigma_{\rm SIDM}$, of dark matter. 
A lot of observational studies focus on using merging galaxy clusters
as they assume the high collisional velocity should increase the chance
of detecting the effects of SIDM.
By assuming galaxies being relatively collisionless $\sigma_{\rm gal} \approx 0$, 
any offset of the DM population from the galaxy provides $\sigma_{\rm SIDM}$ 
relative to $\sigma_{\rm gal} \approx 0~\centi\meter^2$. 
These observational studies include \cite{Markevitch2004} and \cite{Bradac2006b}  
reporting an offset of 25 kpc for the Bullet Cluster;  
\cite{Dawson2013} reporting an offset of 129 kpc and 47 kpc for the southern
and the northern subcluster respectively;
\cite{Jee2015} reporting an offset of 190 kpc for MACSJ1752, and others that we
list in detail in table \ref{tab:offset_results}.
However, other studies using 129 X-ray selected relaxed galaxy groups, 
such as \cite{George2012a} also report offsets of the same order of magnitude,
at around $50 - 150$ kpc. 

Simulators have also joined hands to study the velocity dependence of the
effects of SIDM. There are many staged simulations of mergers of galaxy
clusters that focused on detecting the signal from
SIDM. The `staging' step of these simulations usually involve prescribing a parametric  
spatial distribution of galaxies  
(\citealt{Randall2008d}, \citealt{Kahlhoefer14}, \citealt{Robertson2016}), 
such as an NFW profile, 
and may not simulate the galaxy morphology, nor include dynamical
friction. 
They try to show the level of offsets solely due to SIDM \citep{Kahlhoefer14}  
by initializing the galaxy-DM offset to be zero at the beginning of their 
simulations. Furthermore, these staged simulations commonly use a much higher number of 
galaxy particles than the realistic observable number of galaxies. 
\cite{Randall2008d} found an offset of only 1.8 kpc in the staged merger
simulation with $\sigma_{\rm SIDM} = 0~\centi\meter^2 /$ g using $10^5$ 
galaxy particles. 
When zero impact parameter was initialized for mergers, Kim and Peter et al.
(in prep.), using 5.7k or 57k galaxy tracer particles, also show 
null galaxy-DM offset during most periods of their control staged simulation 
with $\sigma_{\rm
SIDM} = 0~\centi\meter^2 /$ g. While we provide a more in-depth comparison with
these staged simulation in the discussion, we argue 
these staged simulations do not sufficiently probe  
how statistical and observational uncertainties realistically contribute to 
the galaxy-DM offsets. 
As such, when the $\sigma_{\rm
SIDM}$ is increased in the aforementioned staged simulations, the simulators 
can guarantee the offsets are maximally due to SIDM.
When these simulations set the $\sigma_{\rm SIDM}$ to observationally motivated 
levels of $< 3$ \centi\meter$^2 \gram^{-1}$, 
different authors have consistently reported offset signals (50 kpc) 
smaller with some of the more extreme uncertainties estimated from individual observations (100 kpc). 
These simulations have raised questions about how strongly the galaxy-DM offsets 
can constrain the effects of SIDM.
When \cite{Kahlhoefer14} simulated SIDM with both low-momentum-transfer 
self-interaction 
and rare self-interactions of DM with high momentum transfer, they found the maximum 
offsets that are $< 30$ kpc for $\sigma_{\rm SIDM}$ as high as 1.6
\centi\meter$^2$ / \gram.
The reported offset from \cite{Randall2008d}
for $\sigma_{\rm SIDM} = 1.24~\centi\meter^2 / \gram$ is only 53.9 kpc. 
While Kim and Peter et al. (in prep.) found a maximum offset $< 50$ kpc for 
$\sigma_{\rm SIDM} = 3~\centi\meter^2 / \gram$ ,
and \cite{Robertson2016} also found a maximum offset $\lesssim 40$ kpc  
 from a simulation suite of a Bullet Cluster analog 
 with $\sigma_{\rm SIDM} = 1~\centi\meter^2 /$ \gram.

An alternative explanation for most of the observed galaxy-DM offset signal is due to 
statistical and observational uncertainties. Galaxies are
sparse samples of the underlying DM overdensities, it is possible that the 
summary statistic of the sparse sample to be different from those of the 
underlying distribution. It is not clear if there is any physical
origin of the galaxy-DM offset in a CDM universe, 
but any statistical noise leading to an offset can influence this method of 
inferring $\sigma_{\rm SIDM}$. 
Since the Illustris simulation assumes no SIDM, this study is complementary to 
staged simulations for understanding what can contribute to the offsets.
Simply put, we perform a hypothesis test with the galaxy-DM offsets in
the Illustris simulation directly corresponding to our null hypothesis
$\mathcal{H}_0$, with: 
\begin{equation}
\begin{cases}
	\text{the null hypothesis }\mathcal{H}_0: \text{Cold Dark Matter (CDM)} \\
	\text{the alternative hypothesis }\mathcal{H}_1: \text{Self-interacting Dark
	Matter (SIDM)} 
\end{cases}
\end{equation}
and we try to see if the observed offset data can be compatible with offsets
derived from a CDM simulation. 

% This is the first study to compare the magnitude of SIDM offset signal with
% statistical noise, such as projection effects and unknown dynamical history. 
% These latent variables are confounding and can increase the variance of
% the population distribution of the galaxy-DM offsets.  

This exercise is further complicated by the fact that there is no theoretical
foundation showing which observable would be the most sensitive to each
possible type of SIDM. In fact, \cite{Kahlhoefer14} have argued that SIDM 
does not cause
significant offsets between the galaxy and DM peaks, and only leads to an offset
between the corresponding centroids within the dynamical timescale for
relaxation ($\sim$ several Gyr). 
% To access if the offsets we observe is
% completely explained by the method for computing the offset, we investigate   
% the precision to which we can find the galaxy-DM offsets.  
Popular choices for computing the offsets often involve first inferring the summary
statistic of the DM population and those of the galaxy population of a cluster
independently before taking a difference.
While there are well established procedures driven by lensing physics for 
inferring the DM spatial distribution, there is no standard procedure for
mapping the sparse member galaxy distribution. 
We quantify the bias and uncertainty associated with the
statistic for summarizing the member galaxy population. 

Understanding the characteristics of different galaxy
summary statistics of clusters is also important for  
probing the matter fluctuations in the universe. 
One such study is done by performing lensing analyses on the stacked images of 
many small galaxy groups and clusters. 
The derived cluster mass function can provide constraints to cosmological
parameters such as $\sigma_8$. 
For such studies, stacking on the `wrong' centers is a commonly cited
source of uncertainties (\citealt{Johnston2007b},
\citealt{Ford2014}). By comparing the discrepancies of different galaxy
summary statistic, we can find out what can help maximize the lensing signal and 
the possible cause(s) of miscentering. 

% % What are the observational methods for summarizing dark matter distribution?
% % Weak and strong lensing are the most reliable methods for mapping the dark 
% % matter distribution in a galaxy cluster. 
% % Common to all the methods are the estimation of the density peaks. 
% % Uncertainties affect the conclusion for the computation the hypothesis test / parameter
% % estimation
% % Previous work on quantifying galaxy-DM offsets included  
% % What centroids they have used
% % 
% % Physical motivation for using the galaxy density peak 
% % Observation footprints 
% % 
% % Under the assumed Lambda Cold Dark Matter ($\Lambda$CDM) cosmology, it is unclear 
% % that how large the offset $\Delta \vec{s}$ should be. 
% % % Why use simulation data to study the populations? 
% % Other complications for studying galaxy clusters arise from observation
% % limitations. There is not a lot of information that can help constrain the  
% % line-of-sight distance of different components of a cluster. 
 
% % Goals of the paper
% % With the advent of large-scale optical sky surveys, 
% % the number of identified galaxy clusters is growing quickly. 
% % Existing catalogs such as the Abell catalog also contain
% % peaks inferred
% % from the different peak finding methods and the DM peaks. 
% % at least 4000 clusters with at least 30 members. 
% % The future Large Synoptic Sky Survey alone will identify over a hundred thousand galaxy
% % clusters (CITE) in optical wavelengths. 
% % It is important to verify the uncertainties associated with common
% % summary statistics for studying galaxy clusters. Considering the large quantity
% % of data, methods with manual tuning will not scale well. The manual biases may
% % also make it hard to obtain consistent statistics from the samples.

In this paper, we 
1) extract realistic observables from the Illustris simulation for
comparison with observations, 2) explore the pros and cons of the different statistic for 
summarizing {\it the member galaxy population} of a galaxy cluster, 3)	
give estimates for the offsets between the summary statistics of the galaxy  
population and the DM population under $\Lambda$CDM cosmology, which we call 
\begin{equation}
	\Delta \vec{s} \equiv \vec{s}_{\rm gal} - \vec{s}_{\rm DM}.
\end{equation}
where $\bf{s}_{gal}$ and $\bf{s}_{DM}$ are the two-dimensional (2D) spatial
locations of the summary statistic of the galaxy population, and the density
peak of DM respectively. This gives
an estimate of the baseline scatter of offsets without any SIDM. And finally we 
4) examine the properties of the clusters that give outliers in 
the offset distribution and 5) investigate the  
correlations between the physical properties of a cluster and the projected 
observables such as $\Delta s$. 
\begin{figure*}
	\includegraphics[width=\linewidth]{fig1_mass_richness.eps}
	\caption{ {\bf Left figure:} Mass distribution of the group / cluster sized 
		DM halos for different halo selection schemes. Mass estimates obtained by the
		FoF algorithm are labeled as  M$_{\text{FoF}}$.
		Masses centered on the most bound particle within a radius those the 
		average density is 200 or 500 times the critical density of the universe are 
		labeled as M$_{200c}$ and M$_{500c}$ respectively. 
		% Huge discrepancies between the $M_{500c}$ and $M_{\rm FoF}$ of the clusters 
		% indicate the presence of spatially
		% separated substructures for the clusters (See Fig. 
		% \ref{fig:select_peak_visualization}). 
		{\bf Right figure:} 
		Mass-richness relationship of galaxy clusters and groups with 
		$M_{\rm FoF} > 10^{13} M_{\odot}$ assuming different cosmological redshifts
		of the observed clusters. 
\label{fig:mass_richness}}
\end{figure*}

% Basic setup 
The organization of this paper is as follows:
In section \ref{sec:illustris_sim}, we will describe the physical properties of 
the data of the Illustris
simulation (\citealt{Vogelsberger2014}, \citealt{Genel2014a}), 
and the selection criteria that we have employed to ensure that the
quantities that we examine resemble observables but without noise and
systematics from observations. 
Then in section \ref{sec:methods}, 
we explain the methods for computing various 
summary statistics of the spatial distribution of galaxies how we prepare our dark
matter spatial data to resemble convergence maps. We show the statistical performance
of the different summary statistics before we show the main results
in section \ref{sec:results}. In the discussion in section \ref{sec:discussion}, 
we list the implications of our
results and compare it to other simulations and observations. We also 
show how one may make use of the population offset statistical distribution
from the Illustris data to construct a test with 
a null hypothesis of $\sigma_{\rm SIDM} = 0$ and discuss the caveats. 

	Our analysis makes use of the same flat Lambda Cold Dark Matter ($\Lambda$CDM) cosmology
as the Illustris simulation. The relevant cosmological parameters are
$\Omega_\Lambda = 0.7274, \Omega_m = 0.2726$, and $H_0 = 70.4$
km~s$^{-1}$~Mpc$^{-1}$.

\section{The Illustris simulation data} 
\label{sec:illustris_sim}
The Illustris simulation contains some of the most
realistic, simulated galaxies in clusters to date, making it especially suitable for 
verifying the properties of galaxy clusters. We obtained our data from 
snapshot number 135 (cosmological $z=0$) of the Illustris-1 simulation. The Illustris-1
simulation has the highest particle resolution and has incorporated the most 
comprehensive baryonic physics among the different Illustris simulation suites. 
The sophisticated galaxy formation model in Illustris-1 
includes star formation rate, and stellar evolution due to
environmental effects of the intracluster medium, such as ram pressure stripping and
strangulation and feedback from Active Galactic Nuclei (AGN) etc. \citep{Genel2014a}.
The physics of stellar
evolution were solved using a moving mesh code {\sc{AREPO}} \citep{Springel2010}.
The observable properties of galaxies were verified to be statistically consistent
with the Sloan Digital Sky Survey (SDSS) data
\citep{Vogelsberger2014}. 

As the stellar population in Illustris were evolved from the initial condition,
these makes the spatial distribution of galaxies in the Illustris data more 
realistic than galaxies that are prescribed onto DM-only cosmological
simulation data such as those used in \cite{Harvey2013d}.  
Gravitational effects in Illustris-1 have provided realistic dynamics and
spatial distribution of subhalos. The simulated effects include
tidal stripping, dynamical friction and merging. 
Since the profiles of the galaxies clusters were not
provided in symmetrical, parametric forms, we can study 
how asymmetry in the cluster profile affects the estimate of our summary 
statistic. This data allows us to examine cluster galaxies
in a realistic, yet noise-free way. The softening length of the DM particles is
1.4 kpc and those of the stellar particles is 
0.7 kpc, both in constant comoving units \citep{Genel2014a}.

The two sets of data catalogs in use are obtained through two types of halo
finders. The catalog that maps particles to the halo of a certain cluster was 
created by the {\sc{SUBFIND}} algorithm. The friends-of-friends (FoF) 
finder \citep{Davis1985} was further used to identify the affinity
of galaxy-sized halos to a galaxy-cluster. 
These galaxy-size halos are referred to as {\it subhalos} and 
they are the dark matter hosts of what we refer to as galaxies in Illustris-1. 
\cite{Vogelsberger2014a} also extracted the 
absolute magnitude of each subhalo in
the SDSS bands of $g, r, i, z$ as part of the {\sc
SUBFIND} catalog using stellar population synthesis models.

For our analyses, we make use of galaxy clusters / groups 
with at least 50 member galaxies that are within a reasonable observational limit, 
i.e. apparent $i \leq 24.4$ which is the limiting magnitude of the DEIMOS
spectrometer on the Keck telescope when we assume a cosmological redshift of $z = 0.3$
in the $i$ band. The limiting magnitude of 24.2 in the
F814W filter of the Hubble Space Telescope, and the limiting magnitude of the
Canadian-Hawaii French Telescope of 24.5, 
are also close to our chosen limiting magnitude. 
There is relatively large statistical uncertainty if we try
to analyze clusters with less than 50 member galaxies. 
As indicated by the right-hand panel of Fig. \ref{fig:mass_richness}, 
a total of 43 clusters has 
survived this magnitude cut. These simulated galaxy clusters (or groups) have 
masses ranging from $10^{13}$ M$_\odot $ to $10^{14}$ M$_\odot$.  

\begin{figure}
	\centering
	\includegraphics[width=\linewidth]{fig2_color_magnitude_diagram9.eps}
	\caption{Color-magnitude diagram of one of the galaxy clusters that is selected for 
		analysis. This cluster is the 9th most massive. 
		The apparent magnitude is calculated assuming that 
		the cosmological redshift (distance) is $z = 0.3$. 
		We can see a clear overdense region that corresponds to a red-sequence.
		The color-magnitude diagrams of the other clusters can be found in the
		Jupyter notebook at \href{https://github.com/karenyyng/galaxy_DM_offset/blob/master/code/analyses/fig2_color_magnitude_diagram.ipynb}{https://goo.gl/TJmI6s}.
		\label{fig:color_magnitude_diagram}
	} 
\end{figure}

\subsection{Cluster properties}
\label{subsec:cluster_properties}

\subsubsection{Quantifying the dynamical states (relaxedness) of the galaxy clusters}
\label{subsubsec:relaxedness}

Clusters undergo merger activities of a large range of physical scales and 
in the time scale of million of years. 
The dynamical history, or what we call ``unrelaxedness", cannot be directly 
computed from observations.
We quantify the state of the cluster by providing several quantitative
definitions of unrelaxedness and see how they correlate with $\Delta s$.
Some possible definitions of unrelaxedness referred by the simulation community
include:
\begin{itemize}
	\item unrelaxedness$_0$: the ratio of mass outside the dominant dark matter halo over the total mass
		of the galaxy cluster. The lower the ratio, the fewer substructures there
		are in the cluster. 
	\item unrelaxedness$_1$: the distance between the most bound particle from the center of mass as a
		function of R$_{200c}$, the three-dimensional (3D) radius in which the
		average density is 200 times the critical density of the universe. 
		The smaller the distance, there are less asymmetric 
		substructures. 
	% \item [TODO] velocity dispersion from selected galaxies those selection
	% 	criteria will be explained in  
\end{itemize}
To relate these simulation quantities to observation, 
we compute more observation-oriented 
quantities in the method section \ref{subsubsec:KDE}. 

\subsection{Selection of the field-of-view}
\label{sec:FOV}

\begin{landscape}
\begin{table*}
\begin{center}
	%\begin{minipage}{180mm} 
	\caption{ Selection criteria for stellar subhalos (member galaxies) for each
		cluster / group 
\label{tab:member_galaxy_selections}} 

\resizebox{1.35\textwidth}{!}{
	\begin{tabular}{@{}lcccc@{}}
	\hline 
	Data &  Selection strategy  & Sensitivity & Relevant section  \\ \hline
	Field of view (FOV) & FoF halo finder& comparable to FOV of the Subaru
	Suprime camera & \ref{sec:FOV}  \\ 
	Observed filter & $i$-band & consistent among the redder $r, i, z$ bands &   
	\ref{subsec:galaxy_properties}
	\\ 
	Cluster richness  & $i \leq 24.4$ and $z = 0.3$  & sensitive to
	the assumed cosmological redshift of cluster and & \ref{sec:illustris_sim} \\ 
	& & the assumed limiting magnitude of telescope &   \\
	Two-dimensional projections & even HEALPix samples over half a sphere &
	discussed as results  & \ref{subsubsec:projections}\\  
	\hline
	\end{tabular} 
}  % 1.35 textwidth
\end{center} 
\end{table*}
\end{landscape}

We used {\sc{SUBFIND}} for member particle identification of the DM and the 
{\sc{FoF}} finder for subhalo identification.
We understand that this choice of volume selection can be less restrictive than
observational conditions. We make use of this volume selection scheme
for baseline comparisons. Assuming a conservative line-of-sight (los) distance, 
i.e. cosmological redshift, with  $z = 0.3$, 
the projected extent for most of the Illustris galaxy clusters and groups, 
fits inside the field of view of telescopes, such as the Subaru Suprime Camera,
which covers a physical area of $\sim 9$ Mpc $\times 7$ Mpc 
(See \href{https://goo.gl/ClZNvM}{https://goo.gl/ClZNvM} for a Jupyter notebook 
showing the extent of the Dark Matter distribution of the most massive 129
clusters).

\subsubsection{Spatial Projections}
\label{subsubsec:projections}
Unlike in staged simulation, picking out a particular projection for a cluster 
does not always make physical sense.
For highly symmetrical clusters, most projections are similar. 
However, for mergers or asymmetrical clusters, 
there is no obvious choice for the plane of projection that can allow us to
understand the cluster. Depending on the number of merging components, there
may not be any simplistic merger axis that should be 
projected along the plane of the field of view. 

We therefore compute observables based on even sampling of angular orientation 
as our line-of-sight.
As the order of projecting the data and estimating the summary statistic is
non-commutative, we first project the data before estimating any projected 
observable. 
The computation of even angular orientation 
is done by using {\sc HealPy}, which is a {\sc Python} wrapper for
{\sc{HEALPix}} \footnote{HEALPix is
currently hosted at http://healpix.sourceforge.net}
\citep{Gorski2005}. Each line-of-sight centers on a {\sc{HEALPix}} 
pixel.
The number of projections that we employed is 768 for each cluster. With around 
768 projections, the offset distributions of each cluster start to converge to a
stable distribution. 
Even though there are at least 2 identical projections for each cluster due to
one possible line-of-sight from the front and one from the back, it does not
affect any summary statistic. We do not remove the duplication as it breaks
the rotational symmetry in the 2D plane when we try to compute the 2D population
distribution of offsets.  
% Details of the implementation of the projection is available in Appendix
% \ref{app:projections}.


\subsection{Properties of the galaxies in Illustris clusters}
\label{subsec:galaxy_properties}

Different galaxies have different masses, so they should not be considered with equal
importance for peak identification, which requires summing
the mass proxies of different galaxies. One of the most common weighting schemes 
employed for galaxy data is to weight
by the luminosity in a particular band. For some of the methods, we investigate
the differences in peak identification with and without any luminosity weights.
We pick the $i-$band magnitude
associated with each subhalo as the weight. Since the $i-$band is
one of the redder bands, the mass-to-light ratio is not skewed as much due to star
formation activities. 
We further examined if the colors distribution of galaxies in Illustris-1 are
similar to the observed color-magnitude diagrams for clusters.
The Illustris cluster galaxies are realistic enough that it is easy to
identify an overdense region of galaxies known as the red-sequence in the 
color-magnitude diagram such as Fig.
\ref{fig:color_magnitude_diagram}. The red-sequence is prominent even if we
use other colors formed by different combinations of the $r, i, z$ bands.

\section{Methods}\label{sec:methods}
A common and the most precise way of summarizing the DM distribution in a
galaxy cluster is by finding the lensing peaks 
(\citealt{Medezinski2013}, \citealt{Markevitch2004}, \citealt{Zitrin13}).
Additionally, the peak region is physically 
interesting due to the higher particle density and interaction rates. 
The most direct analogous statistic for summarizing the member galaxy
population in a cluster is therefore, also the peak. 
Comparing the DM peak with the summary statistics of the galaxy population that
are not the peak  can have an {\it offset} purely due to the difference in
the choice of the statistic. 
We will state the definition and implementation of the five commonly used 
point statistic or location for summarizing 
the member galaxy population in a galaxy cluster.

We avoid any manual methods for
comparison purposes, scalability and reproducibility. 
Since all the methods listed in this
paper are automated with the source code openly available, 
it is possible for future studies to reuse our code for comparisons. 
% There are a number of decisions ($\sim $[TODO] ADD NUMBER) that one needs to make to 
% determine the summary statistics. We will try to address the sensitivity of the offset
% due to each decision. 
Another major advantage for automation is that it allows us  
to apply
the same methods across the different snapshots of the (Illustris) simulations to
examine the variability of $\Delta s$ across time in future studies. 


\subsubsection{Computing the weighted centroid}
\label{subsubsec:weighted_centroid}
We follow the usual definition of the weighted centroid: 
\begin{equation}
	\bar{\bf x}_w = \frac{\sum_i w_i \vec{x}_i}{\sum_i w_i},
\end{equation}
with $\vec{x}_i$ being the positional vector of each subhalo 
and we use the $i$-band luminosity 
as the weight $w_i$ for the $i$-th galaxy.
Centroids can be biased by subcomponents from merging activities yet the
centroid estimate does not provide explicit evidence for ongoing merger or 
accretion. These estimates are also sensitive to odd boundaries 
of the field of view.

\subsubsection{Cross-validated Kernel Density Estimation (KDE) and the peak finder} 
\label{subsubsec:KDE}
Finding the exact peak of a set of data points 
involves computing the density estimate of the data points and sorting through
the density estimates. A specific version of this density estimation process is
known as histogramming. During the making of a histogram, each data point is
given some weight using a tophat kernel and the weights are summed up at
specific data locations (e.g. $\bf{x}_i$). 
A histogram is not good for peak estimate for {\it sparse} data for two reasons: 1) the
choice of laying down the bin boundaries affects the count in each bin, 2) the choice of
bin width also affects the count in the bin. Only when the available number of data points
for binning is large, the estimates of histograms and smoothed density
estimates are approximately the same. The number of member galaxies ($< 500$) 
is sparse enough for the uncertainty introduced by histogramming to bias our
peak estimate. For the density estimate of galaxy luminosity, 
we adopt a Gaussian kernel. 
The exact choice of the functional form of the smoothing kernel does
not dominate the density estimate as long as the chosen kernel is
smooth \citep{Feigelson2014}. 

For computing the density estimate, the most important parameter of computing 
is the bandwidth of the smoothing kernel, 
which takes the form of a matrix in the 2D case. 
%  We illustrate the choice of kernel width with Fig.
% \ref{fig:bias_variance_tradeoff}. 
When the kernel width is
too large (bottom left panel), the data is over-smoothed, 
resulting in a bias of the peak estimate. On the other hand, when the kernel
width is too small, it results in high variances of the estimate and 
too many peaks due to noise. 
A good illustration can be seen in \citealt{Vanderplas2012} from 
\href{http://www.astroml.org/book\_figures/chapter6/fig\_hist\_to\_kernel.html}{http://goo.gl/jvsfcv}.
The decision of having to balance between creating high
bias or high variance estimates is also known as the bias-variance tradeoff. 
Any other smoothing procedures, including interpolation with splines,
polynomials, and filter convolutions, also face the same tradeoff. 

% \begin{figure*}
% 	% \includegraphics[width=\linewidth]{figN_mass_richness.eps}
% 	\caption{This figure is adapted from \citealt{Vanderplas2012} from
% \href{http://www.astroml.org/book\_figures/chapter6/fig\_hist\_to\_kernel.html}{http://www.astroml.org/book\_figures/chapter6/fig\_hist\_to\_kernel.html}
% under the fair use of the BSD license. \label{fig:bias_variance_tradeoff} }
% \end{figure*}
A well-known way to minimize the fitting error from the density estimate is through
a data-based approach called cross-validation to obtain 
the optimal 2D smoothing
bandwidth matrix ($\Hmat$) of the 2D Gaussian kernel for the
density estimate $\hat{f}$:
\begin{align}
	\hat{f}(\chi; \Hmat) &= \frac{1}{n} \frac{1}{(2\pi)^{d/2}|\Hmat|^{1/2}}
	\sum_{i=1}^n w_i \exp((\chi-{\bf x}_i)^T H^{-1} (\chi-{\bf x}_i)),
	\label{eq:cross_validated_bandwidth}
\end{align}
where the dimensionality is $d=2$ for our projected quantities,
$\chi$ represents the uniform grid points for evaluation, and 
$\bf{x}_i$ contains the spatial coordinates for each of the identified member 
galaxies that survived our brightness cut and $w_i$ is again the $i-$band
luminosity weights for each galaxy.
The idea behind cross-validation is to leave a small fraction of data point 
out as the test set, and use the rest of the data points as 
the training set for computing the estimated density.
Then it is possible to estimate and minimize the asymptotic mean-integrated squared error
(AMISE)  by searching
for the best set of bandwidth matrix values, eliminating any free parameter. 

Specifically, we made use of the smoothed-cross validation \citep{Hall1992} 
bandwidth selector in the statistical package {\sc{ks}} \citep{Duong2007} 
in the {\sc{R}} statistical computing environment \citep{R_core}. 
Among all the different {\sc{R}} packages, {\sc{ks}} is the
only package capable of handling the magnitude weights of the data points 
while inferring the density estimates \citep{Deng2011}. 
Although the particular implementation of KDE has a computational runtime of $O(n^2)$, 
the number of cluster galaxies is
small enough for this method to finish quickly ($\lesssim 0.65$ second per
projection per cluster). 

The resulting KDE contains rich information about the spatial distribution of
the clusters, and we focus on the peak regions.  
We employed both a first and second-order  
finite differencing algorithm to find the local maxima.  
The local maxima were then sorted according to the KDE density in a descending
fashion before we perform peak matching and compute the offset. The exact
procedure is discussed in section \ref{subsec:offsets}. 

For each projection of each cluster, we normalize the density of all 
luminosity peaks to those of the brightest peak. 
Luminosity peaks that sit on top of actual subclusters would then have a density 
comparable to those of the brightest peak. 
Then we sum the density of all the galaxy peaks for a cluster and call this value
$\nu$. When the value of $\nu$ much gets bigger than 1, it indicates the presence 
of projected substructure(s). Even though 
$\nu$ is not expressed in terms of masses, it is a practical measurement
for optical survey data those galaxy magnitudes, but not the spatial mass distribution 
from lensing estimates, are available. 
% Another possible quantification of relaxedness can be computed based on the
% (morphology) ratio of eigenvalues of the galaxy luminosity map bandwidth matrix. 

\subsubsection{Shrinking aperture estimates}

Another popular method among astronomers for finding the peak of a spatial
distribution is what we call the shrinking aperture method.
While we do not endorse this method,
we test if the shrinking aperture method is able to reliably recover the 
peak of the luminosity map.
This method is dependent on the initial diameter and the initial center 
location of the aperture.
This method does not evaluate if the cluster is made up of
several components.
The estimate using the shrinking aperture algorithm can be biased by
substructures. The only way to inform the algorithm about substructures would
be to introduce another parameter to restrict the extent of the aperture, or to
partition the data with another (statistical) algorithm.
More to the point, the convergence of results of this method is unstable. We use a
convergence criterion of having the aperture distance not change more than 2\% 
between successive iterations as a reference. The actual implementation in
{\sc Python} can be found at \href{https://goo.gl/nqxJl8}{https://goo.gl/nqxJl8} while
the pseudo-code can be found in Appendix \ref{app:shrink_apert}.

\subsubsection{Brightest Cluster Galaxies (BCG)}
The BCGs are formed by the merger of many smaller
galaxies. The galaxy-cannibalism makes BCGs typically brighter than the rest of 
the cluster galaxy population by several orders of magnitude. 
However, star formation can cause
less massive galaxies to be brighter in the bluer photometric bands.
To avoid star formation from biasing our algorithm for identifying the
BCG, we find the brightest galaxies in redder bands i.e. the $r, i, z$
bands and found that they give consistent results for all selected clusters. 
We used the $i-$band to pick the BCG for computing the plots and the final results. 
 
\begin{figure*}
	\centering
	\includegraphics[width=.85\linewidth]{fig3_toy_data_mixtures.eps}
	\caption{Comparison of peak finding performances of the shrinking aperture
		peak (shrink peak), KDE peak estimate, and centroid, by
		drawing data points (i.e. 20, 50, 100, 500) from known number of 
		Gaussian mixtures. 
		Panels from the top row contain data drawn from a single Gaussian mixture. The
		panels from the middle row contain data from two 
		 Gaussian mixtures with weight ratio = 7:3. 
	The panels from the bottom row contain data drawn from three Gaussian
	mixtures with weight ratio = 55:35:10. 
	The left column shows how 50 data points drawn from the fixed number of 
	Gaussian mixtures look like. 
	Due to the statistical nature of this exercise, we sampled the data and
	performed the analyses 50 times to
	create the 68\% and 95\% Monte Carlo confidence contours of the estimates in the
	zoomed-in view of the data in the middle
	column. The rightmost column shows how population bias vary as a
	function of the number of drawn data points from the Gaussian mixtures. 
	From the middle and the rightmost
	column, we can tell that the cross-validated KDE peak estimate is the most 
	accurate when
	there is more than one significant component and enough data points ($> 50$).  
		\label{fig:toy_data_mixtures}}
\end{figure*}

\begin{figure*}
	\centering
	% \includegraphics[width=0.85\linewidth]{Fig4_clst1_48_225.png}
	% \includegraphics[width=0.85\linewidth]{Fig4_clst1_48_135.png}
	% \includegraphics[width=0.85\linewidth]{Fig4_clst12_48_135.png}
	\includegraphics[width=0.95\linewidth]{Fig4_clst1_48_225.png}
	\includegraphics[width=0.95\linewidth]{Fig4_clst12_48_135.png}
	\caption{Visualization of clusters (each row is for the same projection
		of the same cluster). Left column: Projected density distribution of DM	
		particle data is shown in orange, with the dense regions in yellow. 
	The identified density peaks are indicated by colored circles. 
		Middle column: The same DM projection after smoothing with a 50 
		kpc smoothing kernel (kernel size is indicated by a white dot on lower right of
		the panel. The thickness of the dot may be larger than 2 kpc
		for the plots on left hand column).
		Right column: Projected galaxy kernel density estimates (KDE) of 
		the $i$-band luminosity map for the member
		galaxies of the same clusters. Each colored contour denotes a 10\% drop 
		in density mass starting from the highest level in red. Each of 
		the magenta ellipse on the
		bottom right corner of each plot show the Gaussian kernel matrix 
		$H$ from eq. (\ref{eq:cross_validated_bandwidth}). 
		The big black 
	circle is centered on the most bound particle as identified by {
		\sc SUBFIND } and the radius of the circle indicates the 
		R$_{\rm 200C}$. The luminosity peaks (square markers) are colored
		by the relative density to the densest peak, the relative density  
		is shown by the color bar.  
		See \href{http://goo.gl/WiDijQ}{http://goo.gl/WiDijQ} 
		and \href{http://goo.gl/89edcM}{http://goo.gl/89edcM} for the 
	visualization of the selected clusters inside two Jupyter notebooks.
		\label{fig:select_peak_visualization}
	}
\end{figure*}

\subsection{Comparison of the methods from Gaussian mixture data}
In order to examine the statistical properties of commonly used point-
estimates of the distribution of the galaxy data, we test them on data drawn 
from Gaussian mixtures with known mean and variance. (See Fig.
\ref{fig:toy_data_mixtures}). The main factors that affect the performance of 
the methods are sensitive to the statistical fluctuations of the drawn data, 
e.g. the
spatial distribution of the data, including 1) the density profile and 2) the
location(s) of subdominant mixtures,
, and 3) the number of data points that we draw.
It is also not enough to just
compare the performance by applying each method for one realization of the
data. We provide the 68\% and the 95\% confidence regions by applying the
each method for many Monte Carlo realizations.
In general, the peaks identified from the KDE density is closer to the 
peak of the dominant mixture (more accurate) than 
both the weighted centroid method and the shrinking aperture method.
For example, in the bottom middle panel of fig. \ref{fig:toy_data_mixtures}, 
it is clear that the green contours
that represents the confidence region for the shrinking aperture peak is
biased due to the substructure, whereas the confidence region for the centroid 
is so biased that it is outside the field of view of that panel.
For right panel of fig. \ref{fig:toy_data_mixtures}, 
we present how the population bias of each method shrinks as the
number of data points increases. For the data generated with more than one mixture, 
KDE peak consistently show less population bias than the shrinking aperture method. 
The performance of the shrinking aperture method fluctuates and is unstable when
the number of data points is increased.

\subsection{Modeling the DM map in Illustris-1 and the lensing kernel}
The most well established method of inferring the projected dark matter spatial 
distribution from observations is through gravitational lensing.
It works by detecting subtle image distortions of background galaxies due to
the foreground dark matter. The resolution of the inferred map therefore 
depends on the properties of the source galaxies that are being lensed, 
such as the projected number density, the intrinsic ellipticities and morphology etc.
To achieve a sufficient signal-to-noise ratio for lensing, 
\cite{Hoag2016} has performed simulation for inferring the optimal bandwidth
for a Gaussian smoothing kernel for the cluster MACSJ0416. 
In the strong lensing regime, \cite{Hoag2016} found a resolution of 11 arcseconds
can best fit the MACSJ0416 data. A kernel bandwidth (this is the standard deviation) 
of 11 arcseconds translates to an angular diameter distance of 50 
kpc assuming a cosmological redshift of $z \approx 0.3$. 
In order to match the resolution of lensing data,
we also employed a smoothing kernel of a similar physical size of 50 kpc.  

To compute the DM spatial distribution from our data, we first make histogram with 2 kpc
$\times$ 2 kpc bin size which is slightly larger than the DM softening length of 1.4 kpc. 
After that, we use a (50, 50) kpc 2D radial Gaussian kernel 
to smooth the DM histogram Illustris DM
particle data. 
There are resolution differences between the smoothed and unsmoothed DM
maps. The unsmoothed histograms tend to show many more local maxima around the major
density peaks (i.e. show high variance). 
The number of DM particles for each cluster is of 
the order of millions and densely packed in the region of
interest. 
Physically, the smoothed histograms of the dark matter of each cluster 
is analogous to a convergence map from a lensing analysis. 


\subsection{Finding the offsets} \label{subsec:offsets}
It is possible to have several peak estimates from the KDE of the member galaxy 
population of a cluster. 
From the density estimate at each peak, we can sort 
the peaks according to their densities. We only match luminosity 
peaks that are at 
least 20\% as dense as the brightest galaxy-luminosity peak to avoid 
computing the offsets of spurious substructures, such as the peaks due to 
small number of galaxies that are located far away from the main concentration of mass.

In general, there are many more DM peaks because there are many more dark 
subhalos than galaxies for each cluster and the resolution of the DM data is
much higher. We write our peak matching algorithm to mimic how humans would
find the closest DM peak to the galaxies peak. 
This is done using a k-dimensional tree (KD-Tree; in our case, k = 2) 
using the densest $n_{\rm DM}$ number of DM peaks:
\begin{equation}
	n_{\rm DM} = \begin{cases}		
		3 \times (n_{gal} + 1) & {\rm if~} n_{gal} < 3 \\
	3 \times n_{gal}  & {\rm if~} n_{gal} \geq 3.
	\end{cases}
	\label{eq:peak_threshold}
\end{equation}
where $n_{gal}$ is the number of significant galaxy peak, and $n_{\rm DM}$
is the number of peaks that went into the construction of the KD-tree.
When there are more
than one dense galaxy peaks located far away from one another, 
the top few densest DM peaks (subhalos) can be located around the same galaxy peak.
i.e. there is no one-to-one matching between the luminosity of galaxies and the
density of detected DM peaks.
Matching purely based on density and luminosity leads to larger offsets.
From inspection of figures similar to Fig. \ref{fig:select_peak_visualization}, using eq. (\ref{eq:peak_threshold}) works well to match the 
appropriate peaks. 
To avoid peak matching from affecting the offset results,
we will only use the offsets with the dominant DM peak for the final results.
We show in the result section that most of the chosen DM peaks
do not have significant deviation from the most gravitationally bound particles.

After identifying the DM peaks, we compute the 
offsets between the matched DM peaks, and the following spatial estimates, including 
\begin{itemize}
	\item the most (gravitationally) bound particle 
	\item the shrinking aperture peaks, the corresponding offset is $\Delta s_{\rm
		shrink}$, 
	\item the number density peaks, the corresponding offset is $\Delta s_{\rm
		num.dens}$, 
	\item the BCGs, the corresponding offset is $\Delta s_{\rm BCG}$ and
\item the luminosity weighted centroid, the corresponding offset is $\Delta s_{\rm BCG}$.
\end{itemize}
Since there can be more than one number density peak from the corresponding KDE
map, we also use a KD-tree to location the closest number density peak to the 
identified DM peak.
% The most bound particle is the location with minimum 
% gravitational potential of the {\sc{Subfind}} identified cluster.
% Due to substructures, it is possible for there to be several minima of similar 
% gravitational potential level. 
After matching the peaks to compute the offsets, 
we report the percentile for the offset distributions.
For instance, the 95\% interval is computed as the narrowest interval that encompasses
95\% of total density (2.5\% of density mass at each end of the tail is
excluded). In case of degeneracy, the interval is also required to cover the 
central location estimate for the distribution.
Additionally, we report other statistic of interest, such as the biweight 
location (analogous to the median)
and the midvariance (robust standard deviation
estimate) to minimize the effects of outliers. 
\citep{Beers90}. We compute the robust statistics using the implementation
from \cite{astropy} as part of {\sc{astropy}}. 

\begin{figure*}
	\begin{center}
	\begin{minipage}[b]{0.45\textwidth}
	\includegraphics[width=0.95\linewidth]{fig8_total_normalized_peak_density.pdf}
	\end{minipage}
	\begin{minipage}[b]{0.45\textwidth}
	\includegraphics[width=0.95\linewidth]{fig7_projected_KDE_offset_distribution.pdf}
	\end{minipage}
	\end{center}

	\caption{Left: A box plot showing the distribution of the total normalized
		peak density ($\nu$)
		for each cluster. Clusters with only one dominant peak has $\nu = 1.0$,
		values bigger than 1 indicate  density fraction contributed by other
		peak(s). 
	  Right: A box plot showing the distribution of $\Delta y_{\rm KDE}$ for each cluster 
		based on 768 projections. 
		The offsets were computed between the closest DM 
		peak to the brightest luminosity peak of each cluster. 		
		The red line of each box shows the median of the projections,
		the box encompasses the 25\% and 75\% percentile of the distribution while
		the whiskers mark the 5\% and the 95\% percentile. The other black crosses
		are data points with extreme values beyond the 5\% and 95\% percentile.
		\label{fig:projected_KDE_offset_distribution}
	}
\end{figure*}


\begin{figure*}
	\begin{center}

	\includegraphics[width=0.65\linewidth]{fig5_symmetrical_1D_pdf.png}
	\caption{ 		
		The smoothed distribution of different offsets of 43 clusters with all 768
		projections. The smoothing bandwidth is determined by Scott's rule for 
		visualization.
		For estimates where several peaks of galaxy data are 
		possible, only the densest peak is matched to the DM peak for computing
		the offsets in this figure. 
		The dark blue area indicates the 68\% density interval
		while the light blue area shows the 95\% density interval. 
		\label{fig:offset_distributions}
	}
\end{center}
\end{figure*}
\subsection{Constructing the hypothesis test} 

% There is a more comprehensive discussion of the representation of offset 
% distributions and the complications due to the different projections of the
% clusters in Appendix [TODO]. 
The representations of the distributions of $\Delta s$ carry
different information and allow different types of statistical tests. 
The most faithful representation of the offsets without any information loss
is:
\begin{equation}
	\Delta {\bf s} = ({\bf x}_{\rm gal} - {\bf x}_{\rm DM}, 
	{\bf y}_{\rm gal} - {\bf y}_{\rm DM} ).
	\label{eq:2D_offsets}
\end{equation}
The PDF of $\Delta {\bf s}$ in eq. \ref{eq:2D_offsets} peaks at (0, 0) when
there is no real offset. It is also possible to do directionality tests,
such as Rayleigh z test, to see if the data points have a preference to land in
a certain direction.
However, when one takes the magnitude of $\Delta {\bf s}$, i.e.:
\begin{equation}
	|\Delta {\bf s}| = \sqrt{({\bf x}_{\rm gal} - {\bf x}_{\rm DM})^2 + 
	({\bf y}_{\rm gal} - {\bf y}_{\rm DM})^2},
	\label{eq:magnitude_offsets}
\end{equation}
the resulting 1D distribution of $|\Delta {\bf s}|$, 
those support being [0, $\infty$),
will not peak at zero even if the original
distribution of $\Delta {\bf s}$ peaks at (0, 0). 
To illustrate how difficult it is to interpret the magnitude values $|\Delta
s|$,  
we can imagine the following transformation.
The values drawn from: 
\begin{equation}
	\left(\begin{array}{c}
			\bf{x}\\
			\bf{y}
		\end{array}
	\right) \sim \mathcal{N}\left(
	\left(
		\begin{array}{c}
			0 \\
			0
		\end{array}
	\right),
	\left(\begin{array}{cc}
		\sigma^2, 0 \\
		0, \sigma^2
	 \end{array}
	\right)
\right),
\end{equation}
will result in $|\Delta s|$ values that follow the Rayleigh distribution:
\begin{equation}
	f(\Delta s | \sigma) = \Delta s /  \sigma^2 \exp(-\Delta s^2 / 2 \sigma^2)
\end{equation}
those peak is at $|\Delta s| = \sigma$, the same standard deviation value of the 2D
Gaussian. Note that this representation  
rules out any probability mass at $|\Delta s|$ = 0 by construction. 
The dependency of $|\Delta s|$ on the parameters of
the 2D distribution is even more
complicated when the the 2D distribution does not approximate a Gaussian 
or when there is more than one peak in the 2D space. 
The shifting of the peak location due to variable transformation 
is seen in the distribution of $|\Delta {\bf s}|$ recorded in table
\ref{tab:offset_distributions}.
For a non-parametrical, asymmetrical 1D distribution due to taking the magnitude, 
finding the narrowest 68\% and 95\% interval
is not a standard statistical procedure and can be more prone to error.

On the other hand, 
the 1D distributions of offsets along a particular spatial axis, 
e.g. $\Delta {\bf x}$ and $\Delta {\bf y}$,
each with a support of $\mathbb{R}$, will not exhibit a discontinuity at zero.
Any shift or asymmetry in the 2D peak location is still obvious. 
The distributions represented by $\Delta x$ or $\Delta y$ 
can be symmetrical so 
the narrowest density interval (aka highest density interval) is easier to find.
Since we have enough samples for there to be
rotational symmetry for the distribution of $(\Delta {\bf x}, \Delta {\bf y})$,
we will show that it does not
matter much if we picked $\Delta {\bf x}$ or $\Delta {\bf y}$ for the 1D representation.
We compute the hypothesis test significance level with the 
 offset $\Delta {\bf x}$ along one of the spatial axes. 
To report the statistics, we also make use of estimates that do not make any
underlying assumption of the shape of the distributions that may skew the
statistical parameter estimate.
In fact, several studies have reported poor single 2D Gaussian fits to 2D offset data
due to the long tails
(\citealt{Zitrin2012a}, \citealt{Oguri2010}).  

In the following sections of this paper, we use $\Delta {\bf s}$  to represent the
two-dimensional offsets, $|\Delta {\bf s}|$ for the magnitude of the offset as calculated
according to the Euclidean distance, and $\Delta x$ or $\Delta y$ to denote the
one-dimensional offset along one of the spatial dimensions.
To compare with observed data, we estimate the 1D spatial components of
the offsets from the merging cluster observations from various sources.   
We make our best attempt to measure $\Delta y_{\rm obs}$, the spatial component of the
observed offset along the axis connecting the subclusters if subcomponents exist. 
In our observed samples, Abell 3827 is the only exception that has no
subclusters but only four bright galaxy peaks in the central region. 
We also show $\Delta x_{\rm obs}$ components, the offset perpendicular to
$\Delta y_{\rm obs}$ for comparison.
For most of the observed offsets, we obtain the estimates from the contour plots 
and descriptions of the corresponding papers. 
For the offsets that are roughly in line with the axis connecting the two subclusters,
we let $\Delta y = |\Delta {\bf s}|$. 
If $\Delta s_{\rm obs}$ is not aligned along the line joining the subclusters,
using $|\Delta s_{\rm obs}|$ instead of $\Delta y_{\rm obs}$ to come 
up with a p-value from the distribution of 
$\Delta y$ will lead to a spurious increase in significance.

The actual significance of each observation is computed by comparing the
observed offset to 
the distribution of $\Delta y$ computed from our data. 
The distributions of $\Delta y$ represent the possible ways that offsets can be observed in a CDM
universe, giving us a rough estimation of the probability 
of seeing the offset from observations under the null hypothesis of CDM 
being true. 
We compute the two-tailed 
p-value from the narrowest density interval (C) of simulated offsets 
that is above the observed values of offsets in the literature, 
i.e. the significance of each observed offset is rounded up to the nearest 68\%, 95\% or 99\%
interval of the corresponding offset distribution from the Illustris data. 
\begin{equation}
	p = 1 - C(|\Delta y| > \Delta y_{obs})
\end{equation}
This underestimates the probability that the
observed offset is compatible with the CDM model so
any disagreement with the CDM model will be more obvious. 


 
\section{Results} 
\label{sec:results}

\begin{landscape}
\begin{table}
	\caption{Robust estimates and the distribution of offsets along the y-axis
			(This is different from the magnitude which has discontinuity at zero).
			\label{tab:p_val_table}}
		\resizebox{1.35\textwidth}{!}{
	\input{Chapters/p_val_table.tex}
	
	}
\end{table}
\end{landscape}

\begin{landscape}
\begin{table*}
	\small
	% \begin{minipage}{180mm} 
	\centering
	 \caption{Observed offsets from clusters with reported evidence of mergers
		 along line connecting two subclusters ($\Delta y$) and the approximate 
		 perpendicular offset ($\Delta$ x).
		 The table mainly contains clusters that have been used to constrain
		 $\sigma_{\rm SIDM}$ using the reported offsets.
	 Any approximate  
	 error estimates are the corresponding 68\% lensing peak uncertainty
	 in the figure(s) of the references, this is due to the lack of uncertainty
	 estimates from the galaxy summary statistics from most literature. 
		Error estimates are omitted when they are not reported by the authors.  
	 All p-value lower bounds are reported by matching to the corresponding method for
	 estimating galaxy summary statistic in table \ref{tab:p_val_table}.
	 \label{tab:offset_results}} 

	\resizebox{1.35\textwidth}{!}{
	 \begin{tabular}{@{}lccccccccc@{}}
	 \hline 
	 Cluster &$\Delta y$ (kpc) & $\Delta x$ (kpc) & $|\Delta s|$ (kpc) & galaxy peak & DM peak &  p-value & subcluster &
	 mass ($10^{14}$ M$_\odot$)  &  reference\\
	 \hline
	 Bullet  & 9  & -23 & 25$~\pm~29$ & num. or lum. & SL \& WL & $ 0.32$ &
	 northwest & 1.5 & \citealt{Randall2008d}\\
	 Baby Bullet & -40&  0 & $\sim 40 \pm \sim 50 $  & lum. & SL \& WL &
	 $ 0.05 $ & northwest & 2.6 &
	 \citealt{Bradac2008}:Fig.4 \\
	 Baby Bullet & 30  & 0 & $\sim 30 \pm \sim 75 $ & lum. & & $
	 0.32 $&
	 southeast
	  & 2.5 & \citealt{Bradac2008}:Fig.4 \\
	 Musketball &  129 & 0 & 129 $\pm \sim 63$ & num. & WL & $0.05$ &
	southern & 3.1 
	 & \citealt{Dawson2013}:Fig.4.7\\
	 Musketball & -47 & 0 & 47 $\pm \sim 50$ & num. & & $0.32$ &
	 northern & 1.7 &  
	  \citealt{Dawson2013}:Fig.4.7\\
		Abell 3827 & 6 & 0 & 6 & BCG & SL & $0.05$ & central & & \citealt{Williams2011a}\\ 
		Abell 520 & 0 & 50& $\sim 50 \pm \sim 50$ & lum. & WL & $ 0.32$ & blue 
		 & 5.7 & \citealt{Clowe2012}:Fig. 4 \\
	 % Abell 1689  & & & & & &  & \citealt{Mohammed2014} \\
		El Gordo &  58 &0 & $\sim 58 \pm \sim 100$ & lum. & WL & $0.05$ & 
		northwest& 11  &\citealt{Jee2014}:Fig.7,8  \\
		El Gordo & 30 & 110 & $ 115 \pm \sim 60$ & num. & & $0.32$ & 
		northwest&   &\citealt{Jee2014}:Fig.7,8  \\
		El Gordo & 6 & 25& $\sim 26 \pm \sim 50$ & lum. & & $0.32$ & northwest & 7.9   
		&\citealt{Jee2014}:Fig.7, 8  \\
		El Gordo & 280 & 280 & 400 $\pm \sim 40$ & num. & & $0.05$ &
		southeast &   &\citealt{Jee2014}:Fig.7, 8  \\
		Sausage &160 & 100& $\sim 190\pm \sim 150$ & num. & WL & $
		0.05$ & north & 11.  &\citealt{Jee2015}:Fig.10\\ 
		Sausage &160 & 160& $ \sim 190\pm \sim 150 $  & num. &  & $
		0.05$ & south & 9.8 & \citealt{Jee2015}:Fig.10\\ 

		Sausage & 320 & 130 & $\sim340 \pm \sim 150 $  & lum. & & $\lesssim
		0.01$ & north & 11. & \citealt{Jee2015}:Fig.10\\ 

		Sausage & 160 & 160 &$\sim230 \pm \sim 150 $  & lum. & & $\lesssim
		0.01$ & south & 9.8 &\citealt{Jee2015}:Fig.10\\ 
	 \hline
	 \end{tabular} 
		}  % textwidth
	 \raggedright{
		 {\it num.} is a short hand for the peak estimate from the number
		 density map. \\
		 {\it lum.} is a short hand for the peak estimate from the luminosity
		 density map, or KDE' in the method description. \\
		 {\it SL} is a short hand for strong lensing. \\
		 {\it WL} is a short hand for weak lensing. \\
	 }

\end{table*}
\end{landscape}




\subsection{The dynamical states (relaxedness) of the clusters}
Out of the $43 \times 768 = 33 024$ projections, $\sim 45\%$ of the projections 
have one dominant luminosity peak and negligible substructures, with the total peak density of the
projection being $\nu \leq 1.2$. 
Another $\sim 50\%$ of the projections have more than one dominant luminosity
peak with $1.2 < \nu < 2.2$. 
Fig. \ref{fig:projected_KDE_offset_distribution} that shows the spread of $\nu$
estimate over different projections per cluster indicates
most clusters show signs of other subdominant peaks.
Visually, the spread of the $\nu$ distribution is indicated by the horizontal 
length of the blue box. 
the median of $\nu$ per cluster are indicated by the red central vertical line
inside each box in Fig. 
\ref{fig:projected_KDE_offset_distribution}.
Only 7 clusters (with ID = 15, 16, 17, 22, 31, 35, 51) out of 43 clusters have $\nu
\lesssim 1.2$ for most of the projections.
Clusters with median values of $\nu > 2.2$ usually have multiple subclusters.
Cluster with ID = 7, for instance, is made up of around 4 disconnected clusters that span
several Mpc.  
There is also a strong correlation of $\sim 0.8$ between $\nu$ and 
each of the two unrelaxedness quantities defined in section 
\ref{subsubsec:relaxedness}. 
This shows that $\nu$ can be a good indicator for the dynamical states of the
cluster.


\subsection{Offset between the matched DM peaks and the corresponding most
bound particle}
There is no significant differences between the matched DM peaks and 
the most gravitationally 
bound particle (hereafter most bound particle).
The median of the offset between the DM peak and the gravitationally bound
particle is (0, 0) kpc. The 75-th percentile of the offsets are at ($\pm2,\pm2$) kpc. 
Most of the other offset values occur below ($\pm 9, \pm 9$). Large offsets
are only seen for clusters with $\nu > 1.2$. The densest DM peak in 3D where
the most bound particle is located, does not
necessarily correspond the densest projected peak in 2D in the presence of 
significant DM substructures.   

\subsection{Offset between galaxy summary statistic and the most bound particle}
As another sanity check, we computed the offsets between different galaxy summary
statistic and the most bound particle. 
Interested reader can refer to table
\ref{tab:most_bound_particle_offset_distributions} for the different
percentile and robust estimates of the distribution of offsets from the most bound 
particle. 
The ranking in terms of increasing distance 
to the most bound particle computed by different method is as follows:
\begin{itemize}
	\item the BCG 
	\item the densest peak of the luminosity map created by weighted the KDE 
		\item the shrinking aperture center from the luminosity weighted galaxy data
		\item the densest peak of the number density map created by the unweighted KDE 
		\item the centroid estimate using luminosity weights, which is a proxy for the
			center of mass
\end{itemize}

In fact, most of the BCG offsets are very small except for two clusters with ID 13
and 33. Both clusters have all the values  of $\nu > 1.5 $ over different projections. 
From the projected density map, we further confirm that
both clusters have significant substructures. It is therefore possible for the
most bound particle to have a similar gravitational potential level as another 
substructure where the BCG is located. 
In general, the offset distributions between the galaxy summary statistics and
the most bound particle have approximately the same level of variance but more
extreme outliers (at the 99\%) than the
offset distribution between the DM peaks and the corresponding galaxy summary
statistics.

\subsection{Galaxy-DM Offset in Illustris}
\subsubsection{The two-dimensional distribution and distribution of $\Delta y$}
The 2D distribution of $\Delta s$ from most methods peak at
around zero ($\lesssim 4$ kpc) with high rotational symmetry. 
The possible sources of offset asymmetry come from clusters with unusual configurations,
those clusters with more distinct, spatially separated subcomponents.
They are the clusters with higher offset variance over different projections. 
Their outlier offsets are in general more extreme than other clusters.
Since there are a limited number of projections,
clusters with such unusual configurations and offset levels 
can leave patterns outside the main concentration of 2D offset data, such as rings. 
This is due to projections that give similar views of the cluster. 
The luminosity weighted centroid offset, i.e. median($\Delta x_{\rm
centroid}'$) = -37 kpc, has the highest asymmetry along x-axis for the peak.
These offsets distributions are recorded in detail in table
\ref{tab:offset_distributions}. 

The method that gives the tightest offset is the BCG, consistent with what we
found in the previous section. 
The 2D offset $\Delta s_{\rm BCG}$ has most of its density located near zero
($\pm 3$ kpc) but 
contains outliers. Having outliers is possible 
as the DM peak is chosen as the closest DM peak to match the
brightest luminosity peak in a particular projection.
See the bottom right panel of \ref{fig:select_peak_visualization}, the BCG 
coincides with the most bound particle. However, the luminosity peak of the
cluster is located at the other mass substructure. 
When there are distantly separated subclusters of similar masses, 
the brightest projected luminosity peak 
may shift from one subcluster to another subcluster between different projections,
while the BCG identification is unchanged between projections.
The aforementioned bias from substructure can be seen when we compare the
offset estimates between the relatively relaxed sample of $\nu < 1.2$ and the
unrelaxed samples $1.2 < \nu < 2.2$. The 99-th percentile increased drastically
from $\pm 19$ kpc to an asymmetrical extreme estimates of $(-684, +1570)$ kpc.
Again, these values are possible because there can be several DM peaks of
similar density due to subclusters located far apart from one another.
The finite number of projections, combined with the substructures, have caused 
the 95-th and 99-th percentile tails of $\Delta y_{\rm BCG}$ of both the full
sample and the unrelaxed sample exhibit noticeable asymmetry, 
but not the relaxed samples.
% Some field of views of the same cluster can be similar and happen to project the
% cluster along its longest extent.
% We do not cut off the outliers nor fit any distribution to the $\Delta y_{\rm
% BCG}$. This is because the tight peak of 3 kpc is probably due to numerical noise, 
% (the softening length is around 1.4 kpc). There may not be a physical reason
% why the BCG center is offset from the DM peak. 
% While we trust our 68\% limit for the BCG for the relaxed sample, 
% the percentile estimates for the 95\% and 99\% seem noisy for the unrelaxed sample.


It is noteworthy that the population spread of $\Delta s$ computed by each method 
differ enough that one needs to know which offset method an astronomical study
used for a fair comparison. 
For the full sample in table \ref{tab:p_val_table} and fig. 
\ref{fig:offset_distributions},
the offsets computed by the peak from the luminosity weighted KDE 
has the second smallest variance. The 68-th percentile of $\Delta y_{\rm
KDE}'$ is at $\pm 25$ kpc. Using shrinking aperture to estimate
the peak location from the luminosity map increased the 68-th percentile of the
offset to more than double those of $\Delta y_{\rm KDE}'$ at $\pm 65$ kpc.
The peak estimate from the number density map has even larger variance, 
with its 68-th percentile being $\pm 84$ kpc. 

Most of the percentile intervals of the unrelaxed samples $ 1.2 < \nu < 2.2$ , 
when compared to the relaxed samples, are around a factor of 2 larger. 
Among the relatively relaxed samples, the variance of the inferred offsets from different
methods still show significant discrepancies. 
The variance of the offset computed from the shrinking aperture method, 
the number density map, and the weighted centroid are still at least a factor of 1.5
larger than those computed using the luminosity-weighted KDE. 
In particular, the 68\% percentile of the centroid method is $\pm 108$ kpc.
This is around one-fourth the typical core radius of massive clusters
\citep{Allen1998}. Our centroid estimates can be more extreme than observations
because we do not restrict the field of view.
 The spread of the offsets inferred by each method affects their ability
for constraining $\sigma_{\rm SIDM}$. We will further elaborate on this point
when we compare our results with staged simulations of SIDM in section
\ref{subsec:SIDM_sim}. 


\subsection{Offset projection uncertainty of each cluster}
When we gather the offsets $\Delta s_{\rm KDE}'$ of the 
768 projections for each cluster,
we can find the offset uncertainty due to projection effects.
The distributed are illustrated in the box plot of Fig. 
\ref{fig:projected_KDE_offset_distribution}. The values of the biweight mid-
variance of $\Delta y_{\rm KDE}'$ for half of the clusters
are $< 23$ kpc. Of the ten clusters (ID = 3, 7, 12, 20, 21, 32, 33, 37, 40 and 46) 
that have mid-variance $ > 40$ kpc, all of them have the median of $\nu > 1.2$.
 
\subsection{Correlations between different variables and the offsets}

Here we investigate a list of physical quantities that have significant to little
correlation with the offsets. 
We use the Pearson product-moment correlation coefficient to quantify linear 
relationship between the pairs of variables
(aka Pearson's r,  hereafter $\rho$).
We describe the significance of the correlation 
based on the p-value reported by {\sc{Scipy}} of seeing the level of 
correlation by chance assuming the pair of 
quantities has no correlation. If the p-value is greater than 0.1, we call the
correlation as insignificant.
As a reference, the correlation of between the 
unrelaxedness criteria defined in section \ref{subsubsec:relaxedness}
for the 43 selected clusters is as high as 0.82. 

Each of the two unrelaxedness criteria has a significant positive correlation of $\sim 0.70$
with the maximum of $\Delta s_{\rm KDE}'$ of each cluster
(hereafter $\max(\Delta s_{\rm KDE}')$).
The offset $\max(\Delta s_{\rm KDE})'$ per cluster also show a high
correlation of 0.77 with the median $\nu$ per cluster (hereafter,
median($\nu$)). The FoF mass of each cluster shows only a slight correlation of 0.28 with 
 median($\nu$).

There is a weak correlation between the richness of the
clusters with $\max(\Delta s_{\rm KDE})$ ($\rho = 0.21$). This weak correlation 
may be due to 
the fact that the peak estimate is only affected strongly by a few bright galaxies near 
the peak. The richness is slightly more strongly correlated with the median of $\nu$ 
with $\rho = 0.33$. 

There is an insignificant negative correlation ($\rho = -0.20$) between the different mass 
measured within a certain density threshold, such as $M_{200C}$, $M_{500C}$, 
and $\max(\Delta s_{\rm KDE})$. The quantities $M_{200C}$ and $M_{500C}$, which
are computed within a shell centered on the most bound particles, have
symmetry assumptions that may not capture the total mass well if there are substructures. 
The FoF mass, which
captures the total mass without any symmetry assumption, correlates positively
and weakly 
($\rho = 0.13$) with $\max(\Delta s_{\rm KDE})$. 

Overall, other than substructures, the offsets do not seem to be affected by
other physical attributes of the clusters.

% We subset the data and visually inspected the samples with the largest 
% galaxy-DM offsets $\Delta s_{\rm KDE}$. We found that ...  
% 
% \subsubsection{Correlations between the offsets and properties of the 
% cluster / groups}
% [TODO] examine the relationship between
% \begin{itemize}
% \item 3D relaxedness
% \item mass 
% \item richness  
% \end{itemize}
 
 
 
\section{Discussion}\label{sec:discussion}

\subsection{How to best visualize the cluster galaxy data?}
We inspected both the luminosity maps and the
number density maps of the member galaxy populations.
With the same selection of bright galaxies of apparent $i$-band$ < 24.4$ at
$z=0.3$, the luminosity maps in general resemble the DM maps more closely than 
the number density maps.
A comparison of the projected 
DM map, the luminosity map and the number density map of 129 clusters 
can be found at \href{https://goo.gl/kZUWrg}{https://goo.gl/kZUWrg}, 
\href{https://goo.gl/R7VNi9}{https://goo.gl/R7VNi9} and
\href{https://goo.gl/lmQUPd}{https://goo.gl/lmQUPd} respectively. 
We encourage our readers to see how it is possible
to create scientifically accurate luminosity contours that resemble 
the DM distribution if the member galaxy
data is of high completeness and purity. The KDE is more than a method for
identifying the luminosity peaks.
(As the high resolution figures need to be downloaded, the 
corresponding Jupyter notebooks may take some time and several refresh of the
web page before they are rendered properly.)

In real observations, missing selection of member galaxies, or 
foreground objects can both affect the inference of the galaxy spatial 
distribution. The number density map can be less susceptible to bias from bright 
foreground objects. It is less clear about the effect of missing member galaxies 
for computing the luminosity map because there is a selection bias favoring 
bright member galaxies.

\subsection{Comparison to merging cluster observations}

The offset distributions in table \ref{tab:p_val_table}
represent an estimation of the spread of $\Delta s$ in a $\Lambda$CDM universe.
While the compilation of the offset distribution suffers 
from the small number of high mass clusters in the Illustris simulation, 
the distribution probes projection uncertainties comprehensively. 
Possible discrepancies that affect our comparison can arise
due to the high purity and completeness of the Illustris galaxy data.
Most observations of massive galaxy clusters with a total mass of M$_\odot
\approx 10^{14}$ have richness $\lesssim 300$, while the most massive Illustris
cluster (M$_{\rm FoF} = 3.23 \times 10^{14} M_\odot$) has a richness of 483.
Furthermore, the Illustris clusters are less massive than the observed clusters
with huge offsets. The Illustris clusters may also provide sufficient samples
of possible spatial configurations of observed clusters.
Despite the possible differences, we try to do a
fair comparison to observation by matching the method of offset inference. 

Other complications can arise from how one choose to subset the offset estimate
based on the cluster properties.
The full sample in table \ref{tab:p_val_table} preserves the cluster mass-abundance
relationship, i.e. each cluster has the same number of projections in
the full sample. However, it also underestimates the offset spread because the
full sample includes $\sim 45\%$ of relatively relaxed projections 
that only have one primary luminosity component.  These clusters would
have been excluded for comparison with bimodal mergers. 
Subsetting with $1.2 < \nu < 2.2$ picks out
cluster projections that are in more similar dynamical states as the observed merging
cluster. 
However, some simulated clusters may have more projections included in this sample
than the other clusters. From inspection of the mass abundance relations in 
Fig. \ref{fig:mass_abundance_distribution}, we found that subsampling with $1.2 <
\nu <2.2$ includes a higher proportion of projections from massive clusters
($\sim 20\%$ more) than 
the full sample. This sampling should not introduce significant bias. 
The observed merging clusters are still more massive than the unrelaxed samples. 

We compile table \ref{tab:offset_results} with the corresponding lower bounds
for the p-values using
the $\Delta y$ distributions of the unrelaxed samples ($1.2 < \nu < 2.2$). 
The p-value lower bounds are reported base on the $\Delta y$ distribution 
from the same methods that the observed values were computed. 
The majority of the p-values from table \ref{tab:offset_results}, 13 out of 15, are
$ > 0.05$, only 2 p-values are below 0.05. 
This means the observations are mainly consistent with the null hypothesis: 
It is possible to see offset values as extreme as reported by observations
in a CDM universe. 
However, this does not mean that the CDM model is more probable than the SIDM model. 
We will discuss the full physical implication of this results in section  
\ref{subsec:limitation_of_pvalue} after comparing our results to the SIDM signal 
in section \ref{subsec:SIDM_sim}. 
We continue to discuss the possible use of table \ref{tab:p_val_table} and
the p-values in tables \ref{tab:offset_results}.

To avoid any report from this study directly quoting a p-value, we did not
combine the p-values from different observations. 
This is because the computation of p-value does not fully
take the true uncertainties of the observations into account. 
Most of the quoted uncertainties from \ref{tab:offset_results} are only 
from the lensing estimate, but not the error estimates of the galaxy summary 
statistic. 
Any future studies that wish to claim significance or rejection based on
comparison to a simulation will 
need to carefully track down the contribution of uncertainties from each aspect
of the cluster analysis.  
It may be reasonable to inversely weight offsets by the 
reported bootstrapped uncertainty $|\Delta s_{\rm KDE}|$ for each cluster.

As warned by the American Statistical Association in a statement \citep{Wasserstein2016},
p-value can be misinterpreted easily, so
we must take caution in the procedure of computing and interpreting the p-values. 
For example, the high energy physics community deliberately set the detection
threshold to be 5 $\sigma$ to account for how systematic and bootstrapped 
uncertainties can be
underestimated or unaccounted for. 
If the reliability of each observation is 
approximately the same, averaging the p-values can be one way of combining the p-values, 
but there is no consensus on the best practice of combining the
p-values. The rule used to combine the p-values 
needs to be carefully designed according to the goal of the experiment.
When one keeps on including more observations,  
there will be some observations with extreme p-values that greatly influence the
overall p-value. It is necessary to adjust for the sample size for such running
experiments.   
There are discussions of how to set stopping rules \citep{Demortier2007}
to account for the sample size to determine what p-value level should indicate meaningful
discrepancies from the expectation of the null hypothesis {\it before}
analyzing the data. 
For this analysis, there is a long list of choices that can be up for debate, 
such as, whether to count the sample 
size base on each offset value or each cluster, whether to count all the
observations of the same cluster (e.g. both \citealt{Williams2011a} and
\citealt{Massey2015} have studied Abell 3827), or whether to count both $\Delta x$
and $\Delta y$ or just the direction along the approximate merger axis, and the
list goes on.
Regardless, computing a combined p-value for all the listed observations alone 
does not help understand how to best constrain SIDM.
We instead focus on discussing the implications of the large uncertainties for
constraining SIDM in section \ref{subsec:limitation_of_pvalue}. 

\subsection{Comparison of the offset results to studies of clusters and groups}
Our results showing offset distribution are highly relevant to two other types of
observational studies: 
The first type tries to map the DM distribution of galaxy clusters 
using lensing techniques. 
The second type focuses on finding the best way to stack galaxy groups for
inferring cosmological parameters from the galaxy cluster mass function.
In the following section, we will show that cosmological simulations generally show 
the BCGs has the tightest offsets from the DM peaks, which is consistent with
our finding.
Observational studies, however, in general do not find offset as tight as the 
simulations. We discuss some factors (other than SIDM) quoted in the literature that 
have shown to affect the observed offsets, including the lensing resolution and the
choice of the summary statistic for the member galaxy population. In particular, 
in the comparison of the luminosity peak to other galaxy summary statistic,
the luminosity peak shows promise to give the second least amount of bias,
after the BCG. 
 
To establish the baseline of the tightest $\Delta s$, we first discuss and 
compare the $\Delta s_{\rm BCG}$ constraints.   
\cite{Cui2015}, using 184 galaxy clusters with M$ > 10^{14}$ M$_\odot$ in an
N-body and hydrodynamical cosmological simulation suite powered by GADGET-3, 
% simulation suite with two types baryonic feedback prescription models,
% including radiative cooling, start formation and supernovae feedback, but one with feedback from Active
% Galactic Nuclei (AGN) and one without, \cite{Cui2015} 
also identified the maximum smoothed particle hydrodynamic (SPH) density peak
to summarize 
the galaxy population in their cluster samples. \cite{Cui2015} found
the majority of offsets between BCGs and the most gravitationally bound particle to be
mostly below 10 $h^{-1}$ kpc. They also reported some extreme outliers 
spanning up to several hundred $h^{-1}$ kpc due to the disturbed cluster morphology. Our 
tight 68-th percentile of 
$\Delta y_{\rm BCG}$ at $ \pm 3$ kpc gives some confidence that 
we have identified most of the BCGs correctly in the Illustris simulation.

The distributions of $\Delta s_{\rm BCG}$ derived from simulations, in general, 
are less spread out than those computed in most observational studies.
For example, \cite{Oguri2010} have analyzed 25 X-ray luminous 
massive galaxy clusters of the LoCuSS survey. 
Observations were performed with a large FOV ($\sim (3 h^{-1}$ Mpc$)^2$) 
using the Subaru Suprime Camera. 
By fitting elliptical NFW models to the weak lensing data, \cite{Oguri2010}
showed a long tail
 distribution for $\Delta s_{\rm BCG}$, which they fit with two 2D Gaussians.
The first tighter 2D Gaussian had a standard deviation being 90
$h^{-1}$ kpc for describing the
offset for most clusters, the long tail spans around 1 Mpc was fit by a second 2D
Gaussian with a standard deviation of 420 $h^{-1}$ kpc. This second component
in the tail region contains $\sim$ 10\% of the clusters in the study and is
consistent with the number of extreme outliers that we have.   
Note that by subsetting to only consider relatively relaxed clusters, 
there is a tight maximum bound $\pm$19 kpc for $\Delta y_{\rm BCG}$ in our study. 

One major source of uncertainties for computing $\Delta s_{\rm BCG}$ is the
misidentification of BCG.
To see the effects of misidentifications of 
BCGs, some studies have made use of N-body hydrodynamical cosmological
simulations to compute the 2D distances between the BCG and 
the second most massive galaxies. \cite{Johnston2007b} and 
\cite{Hilbert2010} (using the Millenium simulation)  found 
the one sigma level offsets for misidentified BCGs at 380 $h^{-1}$ kpc, and 410 
$h^{-1}$ kpc respectively. This is consistent with the 95-th percentile of the
unrelaxed and the full sample of the BCG in our study and also the tail of the
$|\Delta s_{\rm BCG}|$ for \cite{Cui2015}. If one wishes to use BCG with high
confidence, it may be necessary to set a stringent standard of the morphological
characteristics such as using the large half light radius for classifying a BCG.

Another source of uncertainty for $\Delta s_{\rm BCG}$ was from lensing. 
\cite{Dietrich2012} performed an analogous analysis of the work of \cite{Oguri2010} 
using the N-body Millenium Run (MR) simulation.
They showed that a combination of shape noise and modeling choices 
alone can lead to hundred-kpc-level offsets between the most bound particle 
(a proxy of the BCG) and the lensing peak for cluster-sized DM halos 
(M $> 10^{14 }M_\odot$).  
\cite{Dietrich2012} ray-traced through the DM substructures in the MR simulation 
as mock lensing observation of 512 clusters.  
Without any smoothing and shape noise, \cite{Dietrich2012} showed 
90\% of the lensing peak and the 
most bound particle, which is a proxy for the BCG, agree to 2.0 $h^{-1}$ kpc
(0.65 arcsec at z = 0.3). 
With shape noise,
even at the source galaxy density of space-based quality optical data of n = 80
arcmin$^{-1}$, fitting NFW halos and using the center as the DM gave a 
distribution of $|\Delta s_{\rm BCG}|$ those mode is at around 9 arcsecs (this is
approximately the 1 standard deviation estimate in 2D). 
Lowering the source galaxy density to 30 arcmin$^{-2}$ increased the mode
to 22 arcsec with a 95-th percentile at 85 arcsec 
($\sim 90$ kpc and $\sim 400$ kpc at z = 0.3, comparable to \citealt{Oguri2010}). 
Smoothing, in the presence of shape noise,
resulted in an offset distribution with the mode at
15 arcsec ($\sim 60$ kpc at z = 0.3). 
The offsets do not simply depend on the smoothing bandwidth, 
but also the number density of the source galaxies that are lensed. 
While the uncertainty from smoothing the DM map is
relatively unimportant in the Illustris analysis due to the much higher resolution 
of the DM particles than the sparser source galaxies used for lensing, it
highlights why the bootstrapped uncertainties from the observed
DM peak need to be accounted for during the comparison between the
Illustris results and the observations.

It is noteworthy from the work of \cite{Dietrich2012} that 
smoothing alone can cause the peak offset to shift from several arcsecs to
around 1 arcmin.
Any other morphological features from the smoothed DM map will be subject 
to uncertainty of similar order of magnitude as the peak
estimate, but with a lower signal. 
It is therefore hard to use other morphological features, such as tail or shape 
of a peak,
as suggested by \cite{Kahlhoefer14} for constraining SIDM.
There is a risk of seeing substructures that mimic the sought-after 
morphological patterns. After all, other cosmological experiments 
that have shown
that it is possible to detect patterns once scientists look hard enough, 
such as the initials of Stephen
Hawking in the seventh-year Wilkinson Microwave Anisotropy Probe (WMAP) 
Cosmic Microwave Background data \citep{Bennett2011}.

Now we turn to exploring complementary methods for summarizing
the galaxy population of a cluster, and show that  the luminosity peak 
is the second best choice than the BCG for summarizing the galaxy statistic.
A unique BCG does not always exist for a cluster (or the subcluster).   
Bright galaxies in the dense region of the cluster are the possible progenitors 
of the BCG and therefore a reasonable choice when there is no unique BCG. 
\cite{George2012a}, for example, examined 129 X-ray selected non-merging galaxy 
groups in the COSMOS field.
They found that around 20\% to 30\% of groups have non-negligible discrepancies
between different galaxy centroids. 
By stacking on a bright galaxy near the X-ray centroid, they found  
the resulting lensing strength is higher than the stacked lensing signal based
on other galaxy centroids, including the BCG. 
For groups with clear BCG candidate, \cite{George2012a} gave the range of
offset between the BCG and the assumed halo center as $\lesssim 75$ kpc. 
The KDE peaks from the luminosity maps of the Illustris samples show a much 
tighter offset to the 
lensing center than any other centroids that \cite{George2012a} investigated. 
The weighted or unweighted centroid measurement from \cite{George2012a} has a 
$|\Delta s|$ with standard deviation at 50 - 150 kpc from the
lensing center with long tails (of around several hundred kpc). 
In comparison, the median (26 kpc), mean (37 kpc), standard deviation (35 kpc) 
and 75-th percentile (49 kpc) of 
$|\Delta s_{\rm KDE}|$ from all the Illustris samples are below 50 kpc. 

We attribute the small amount of population bias of $\Delta s_{\rm KDE}$ in our
analysis due 
to cross-validation, a procedure that is commonly seen and well accepted 
in the top journals 
from the statistics and the machine learning communities. 
Not only does the algorithm help
determine the eigenvalues, but also the optimal eigenvector direction of 
the bandwidth matrix. 
Most literature does not discuss how the bandwidth of 
the kernel for the smoothed maps are determined.  
It is unclear if such results will enjoy the same accuracy level as what
we demonstrated. 
If scientists hand tune the smoothing bandwidth, it is hard to
avoid setting the bandwidth to fit the preconception of how the density
contours of the cluster 
should look like, and inadvertently biasing $\Delta s$.

% [This study is too weird to be included - the FOV is unusually small]
% The feasibility of using luminosity peak as a complementary method to represent
% the galaxy population has also been verified in observations.
% Using 10 000 Sloan Digital Sky Survey (SDSS) galaxy groups and clusters 
% in the Gaussian Mixture Brightest Cluster Galaxy (GMBCG) catalog \citep{Hao2009},
% \cite{Zitrin2012} were able to use the smoothed luminosity maps of the red-sequence 
% galaxies to locate the BCG, which the authors assumed to be the proxy of the DM peak. 
% The richness of the groups range from tens of members to around 160 members. 
% After optimizing the different degrees of smoothing polynomials and 
% assuming a power-law spatial distribution of galaxies, \cite{Zitrin2012} found
% a 24-th degree spline interpolation to best fit the galaxy data within a small
% assumed FOV of
% 120 arcsec$^{2}$  for a luminosity peak. (This FOV translates to 
% $\sim$ 110 $h^{-1}$ kpc$^2$ for the redshift range in the
% paper \citealt{Zitrin2012}). The densest luminosity peak from the spline interpolation had 
% a very close agreement with the identified BCG those mode is $\sim$ 4.2
% $h^{-1}$ kpc after removing 12\% of outliners those offsets are about as wide
% as the FOV. 

% There are also possible differences in the accuracy of our peak
% estimation method and other studies. 
% The population mid-variance of $\Delta y_{\rm KDE}$, which quantifies the
% population bias from the DM peak estimation, is 23 kpc (the 68-th percentile is
% $\pm$ 25 kpc).

% Finally, we try to conclude this subsection by discussing how the offset 
% information of different summary statistic of a cluster can help center the cluster for
% stacked lensing analysis. 
% The equation for modeling the density for stacked clusters usually 
% involve a density term for the BCG,
% one halo term centered on the BCG, some miscentering correction, and a nearby
% halo term \citep{Johnston2007b}. 
% When the galaxy summary statistics disagree (offset $> 50$ kpc),  
% the BCG (and luminosity peak) often land on top of one of the dense subcomponents
% of a generally asymmetrical spatial distribution.
% The weighted centroids,  
% which are the proxies for the center of mass for these disturbed clusters,
% on the other hand, often land in between dense substructures.
% For such an asymmetrical cluster, 
% there is no point location that will simultaneously satisfy the
% requirements of 1) maximizing the density, and 2) preserving symmetry.
% If a high precision is required, one can do it at the sacrifice of a smaller sample size. 
% One can only stack clusters those galaxy summary statistics, e.g.
% the centroid and BCG (or the luminosity peak), have high degree of agreement.
% To take care of the complex morphology for clusters with multiple components,
% a more sophisticated stacking model is needed.
 
  
\subsection{Comparison to staged simulations with SIDM}
\label{subsec:SIDM_sim}
Staged simulations are controlled experiments for probing the contribution
of SIDM to offsets in mergers of cluster components. 
The non-deterministic nature of particle interactions means that it is not easy
to predict the offsets analytically.
Furthermore, there is no consensus on what type of clusters might best show the
effects of SIDM, which depend on the model of the SIDM.
Currently, the studies of SIDM have been restricted to those with isotropic
scattering and velocity-independent cross sections.   
The two main classes of SIDM that are studied include those with 
frequent, short-range interactions that can be broadly be modeled by an
effective drag force, and those with rare, longer range interactions those
effects are not well approximated by a drag-force \citep{Kahlhoefer14}. 
In addition having a dependence on the merger configurations, the offset due to
SIDM also has a time-dependence.    

At SIDM cross section level favored by current literature $\sigma_{\rm SIDM}
\lesssim 1~\centi\meter ~\gram^{-1}$, a list of SIDM simulation studies (Kim \&
Peter 2016, \citealt{Robertson2016}, \citealt{Kahlhoefer14}, \citealt{Randall2008d})
have reported that, during the offset is observable, 
offset generally scales linearly with $\sigma_{\rm SIDM}$,  
up to an offset of 40 kpc.  Increasing the cross section to $\sigma_{\rm SIDM} = 3
~\centi\meter / \gram$ only increased the maximum offset to approximately 50 kpc.
For simulations of major mergers of two subclusters, it takes time for the self-interaction of DM
to manifest and lag behind the galaxies. Kim \& Peter (2016), for example, showed that
this lag starts to be observable approximately after the subclusters reach apocenter. 
While the offset may persist when the subclusters are returning for a
second collision, the magnitude of the offset can fluctuate over this period.
In general, the offset is affected by the phase of major mergers that cannot be
directly calculated from observations. Some of the best estimates of merger
phase from observations
can only give 1-sigma uncertainties up
to 1 Gyr precision (\citealt{D13}, \citealt{Ng2014}).

This small level of offset due to SIDM favors the use of statistic 
with less uncertainty for computing SIDM constraints. 
The uncertainties computed from the number density peak, centroid and shrinking
aperture method simply overwhelm the signal. 
A offset level of 50 kpc is within the one-sigma level of $\Delta s_{\rm num.
dens}$, $\Delta s_{\rm shrink}$, and $\Delta s_{\rm centroid}$, and the two-sigma
level of the $\Delta s_{\rm KDE}$. 
Even though $\Delta s_{\rm BCG}$ has tight distribution for relaxed clusters, misidentification of the BCG 
may increase the tail of the distribution to render it unsuitable to be used 
for secure constraints. To sample the tight distribution of $\Delta s_{\rm
BCG}$ at a higher
resolution than provided here, a simulation at galactic scales with realistic
baryonic feedback will be needed. 

Our results also illustrate one main difficulty for inferring $\sigma_{\rm
SIDM}$, it is hard to propagate and characterize observational error  
for the computation of $\sigma_{\rm SIDM}$. Kim \& Peter (2016) have shown 
that the SIDM offsets depend on the time of the 
merger, impact parameters, collisional velocity and mass concentration. 
Yet, there is no analytical model that includes all these merger
parameters in the calculation of $\sigma_{\rm SIDM}$.  
The alternative way of estimating $\sigma_{\rm SIDM}$ is to compare the observed
offset to staged SIDM simulations.
However, SIDM simulations treating all the observed offset signal of a cluster 
to be contributed by SIDM can give a biased estimate.  
As we have demonstrated, $\sigma_{\rm SIDM} = 0$ does not mean zero offset for
individual clusters.  
Only the population estimate of offsets in a CDM universe, i.e. the mean,
gives zero offset.  
At this point, it is unclear that the analysis with a population of
merging clusters, can provide an excess in population offset signal at a
characteristic scale above the noise level, and give 
better estimates for $\sigma_{\rm
SIDM}$ than the study of individual cluster.  


\subsection{Prospect of detecting SIDM with confidence based on our results} 
\label{subsec:limitation_of_pvalue}
  
So far our analysis has focused on two comparisons: 1) comparing $\Delta s$ from
the Illustris simulation to the observed values and 2) comparing $\Delta s$ to  
the pure SIDM offset signal from staged simulations of SIDM. 
These comparisons informed us that the SIDM signal ($< 50$ kpc) produced by the most favored
$\sigma_{\rm SIDM}$ is less consistent with the higher end of the observed offset values ($\sim
100$ kpc) than the offset uncertainties in the Illustris CDM simulation (68-th 
percentile ranging from 3 to 200 kpc depending on the galaxy summary statistic).
However, we must emphasize these comparisons are
insufficient to rule out small SIDM cross section.  

Realistic mock observations of SIDM (cosmological) simulations 
with small cross section may match observations in better or worse ways. 
It is also possible that the offset levels generated by clusters with small $\sigma_{\rm SIDM}$ 
to be indistinguishable from our results. 
SIDM with a small cross section only produces offset under restricted set of 
merger configurations (Kim \& Peter 2016): 
massive clusters with relatively low merger velocities, small impact
parameters, and large halo concentrations. Yet the offset may only last
for below 5 Gyrs with fluctuating magnitudes.  
Most clusters, therefore, do not have much constraining power on SIDM.
Even in cosmological simulation of SIDM, one would need a selective observational strategy
before the SIDM have statistically significant offset signal when compared 
to the corresponding population of clusters in a CDM simulation. 
Under such restrictive requirements, the Illustris simulation may not contain 
many clusters for such a comparison study.  A bigger practical
concern is the forbidding computational cost of a cosmological simulation with SIDM.
 
 
\subsection{Possible improvements to better understand clusters}
The data in the Illustris simulation has proven to be an excellent source of
understanding observational uncertainties of galaxy clusters.
Further work can be done to make mock observations more realistic with some of
the following improvements:
\begin{itemize}
		\item adding foreground structures 
		\item improving the peak detection threshold from the KDE luminosity map based on
			the flux per area at the peak location and compare that to observational
			limitations
		\item examining the effects of incompleteness by removing some member galaxies randomly 
		\item computing the velocity dispersion of the galaxies along different
			line-of-sight to see how the velocity dispersion correlates the different
			measures of unrelaxedness for different projections
		\item testing the selection of member galaxies using the red sequence
	\end{itemize}


\section{Summary}
We have shown that 
\begin{itemize}
		\item it is possible to  
			see $\Delta s$ as extreme as those in observed merging galaxy clusters assuming that
			$\Lambda$CDM is the true underlying physical model.  \\

		\item the contribution of statistical uncertainty to the galaxy-DM offsets 
			for $\Lambda$CDM clusters is {\it not} negligible when compared to the reported
			levels of offset from staged SIDM simulations ($\sim 50$ kpc).\\ 

		\item only the galaxy-DM offsets derived from BCG and the weighted peak
			derived from the cross-validated
			luminosity map have two-sigma uncertainty levels within the reported SIDM offset
			level ($< 50$ kpc). Other methods have one-sigma uncertainty levels that overwhelm the
			SIDM offset signal.\\ 

		\item while the location estimates of the 2D spatial distribution of offsets and the
			1D spatial distributions of the offsets ($\Delta y$) in the Illustris
			simulation are approximately zero.
			The root-mean-square of the magnitude of the
		offsets, $|\Delta s_{\rm KDE}|'$ is $\sim 30$ kpc from the Illustris sample. 
			does not necessarily map to zero magnitude of offset.  
			Any studies that use a linear $\sigma_{\rm SIDM}$-offset relationship
			that starts with $\Delta s = 0$ to $\sigma_{\rm SIDM} = 0$ can be biased. 
			Since the uncertainty does not always bias the offset high, 
		the following scenario may be possible: zero magnitude of
			$\Delta s$ from merging clusters 
			can be consistent with a range of models with small
			$\sigma_{\rm SIDM}$ under realistic observational conditions,
			but this hypothesis cannot be probed by the Illustris data.\\
  
	 \item it is unclear if using a population of merging galaxy clusters 
		 can generate an excess of population offset signal at any characteristic
		 scale above the statistical
		 and observational noise level as estimated by our study.   
			\\


		\item the locations of the most gravitationally bound particle are consistent 
			with the BCG for systems with little substructures.  \\

		\item the identified BCG can have a large offset from the DM peak due to a
			combination of effects from substructures and projection. \\

		\item the BCG has the smallest offset to the dominant DM peak.  
			The KDE peak of the luminosity map after careful cross-validation 
			gives the second tightest offsets from the DM peak.  \\
		
		\item a naive implementation of the shrinking aperture is easily affected 
			by substructures even for clusters with one
			dominant component. \\  

		\item with high completeness and purity of member galaxy data, the
			luminosity map produced by a cross-validated kernel density estimate 
			resembles the DM spatial distribution more closely than 
			the number density map of member galaxies. The significant peaks from the 
			luminosity map can also be used to characterize the amount of substructures in
			the cluster.\\ 

	
\end{itemize}


\section{Acknowledgements}
% Our software setup is available through a Docker image on DockerHub while the
% The main code are version controlled via Git and GitHub. 
Karen Ng would like to thank Professor Thomas Lee for the helpful discussion of 
the construction of the p-value hypothesis test. 
Part of the work before the conception of this paper was discussed during 
the AstroHack week 2014. This work made used of {\sc IPython}
\citep{Perez2007}.
Part of this work was performed under HST grant (TODO ask Dave for grant
number). 
% 
% 
% 
% % alternative ways of characterizing differences between DM and galaxy
% % distributions.
% 
% % \section{The Physical properties of a galaxy cluster}
% % State variables are missing.
% % A galaxy cluster is not a closed system. 
% 
% % correct for multiple comparisons
% % High number of latent variables 
% % have to rule out the possibility that observables of SIDM being affected by
% % the merger history of the cluster. 
% 
% 

