% The following commands produce page numbering at the bottom
% center using roman numerals per UC Davis requirements for the
% front matter of the dissertation:
% \pagenumbering{roman}
% \pagestyle{plain}
% The following command produces double-spaced lines for the
% remainder of the document:
\doublespacing

\setcounter{chapter}{4}
\chapter{Conclusions}{}{}
\label{chapter5}

% \epigraph{``Every great and deep difficulty bears in itself its own solution. It forces
% us to change our thinking in order to find it."}{Niels Bohr}
\epigraph{``Information is the resolution of uncertainty."}{Claude Shannon}


\section{Overview}
The scientific method is about verifying theories with evidence.
With the low signal-to-noise of the astrophysical systems that I study, 
there are layers of complexity that we must quantify and understand
before we can compare our theories against our data.
To this end, my work has employed several probabilistic  
methods for representing both the physical signal and the associated uncertainties. 
Here, I give a summary and the future prospect of each of my three projects.   

% My dissertation is about the analysis and the modeling of
% large-scale astrophysical systems. 

% \section{Summaries and implications of my projects}
\subsection{Modeling merging galaxy clusters: a case study with El
Gordo}
The first study that I presented was a Monte Carlo simulation of El Gordo.
By drawing the distributions of the observables,
we evolved the motion of the merger of El Gordo analytically
and gave estimates of the latent variables. 
We (re)discovered how the qualitative spatial location 
of cluster components, such as the cool core 
and the radio relic, can be used to construct   
broad quantitative prior distributions for 
constraining the merger phase.  
We showed that it is not only possible, 
but more likely that the merging galaxy cluster, El Gordo, is in a return phase
than an outgoing phase.       
 
For future studies of galaxy clusters, 
it is desirable to obtain a more quantitative version of the 
prior information so a more robust statistical framework can be built for model
comparison. 
I advocate that simulation studies of merging galaxy cluster  
to 1) make the simulation data available for reanalysis, or 2) if there is
enough number of galaxy clusters, provide the
(joint or conditional) probability density function\footnote{This is {\it not}
the same as a histogram. Histograms are lossy representations of the information 
available. See Chapter 3 for a discussion of why a kernel density estimate or
other regression methods should be used instead.} (PDF) of the physical quantities
of galaxy clusters. 

\subsection{The offsets between the member galaxies and the dark
matter in clusters: a test with the Illustris simulation}
A second related but more general analysis of galaxy clusters was done using the
cosmological simulation, the Illustris simulation. 
During the analysis, we found the best 
representations of the galaxy populations of galaxy clusters. 
The Brightest Cluster
Galaxy (BCG) and the peak from a cross-validated kernel density estimate (KDE)
can give high precision estimates of the DM peak to within $\lesssim$ 30 kpc.
Other methods such as the shrinking aperture peak estimate and the peak estimate 
from the number density have large one-sigma uncertainties $\sim 80$ kpc.   
Knowing the best summary statistic can help pin down the uncertainties from  
the offset between the galaxy population and the dark matter (DM). Furthermore,
the KDE peak can be used as a proxy for locating the DM peak for stacked
lensing analysis.  

The second main finding of the analysis was from the hypothesis test that we
conducted.
It is possible to compute the spatial 
offset between the DM and the galaxies as 
\begin{equation}
	\offset =  {\bf s}_{\rm gal} - {\bf s}_{\rm DM}
\end{equation}
using ${\bf s}_{\rm gal}$ a statistic for representing the galaxy population,
and ${\bf s}_{\rm DM}$ the peak of the DM spatial distribution.
We can further decompose the observed spatial offsets according to the different
contributions: 
\begin{equation}
	\offset_{\rm obs} = \offset_{\SIDM} + \offset_{\rm dyn} + n + \cdots
\end{equation}
where $\offset_{\SIDM}$ is the offset due to the self-interaction of dark matter, 
$\offset_{\rm dyn}$ is the offset due to dynamical friction, $n$ is the
contribution from noise and measurement uncertainties, while ``$\cdots$" denotes
any other possible contributions that we have not accounted for. 
We have shown that $\offset_{\rm obs} \approx  n  > \offset_{\SIDM}$, i.e. 
statistical noise can match observations of merging galaxy clusters, better than 
the self-interaction of DM can match observations. 
The observed offset levels are possible even under  
a Lambda Cold Dark Matter ($\Lambda$CDM) cosmology. 

To better estimate $\sigmaSIDM$, it is necessary to perform a Bayesian model 
comparison. This requires computing the Bayes factor, which consists of the
posterior ratio of different DM models:
\begin{equation}
	\tau = \frac{Pr(\sigmaSIDM = \sigma| \offset_{\rm obs})}{Pr(\sigmaSIDM = 0|
		\offset_{\rm obs})}.
\end{equation}
From our analysis of the Illustris simulation, we have provided a better idea of how to evaluate 
$\offset$ to minimize the contribution from $n$. 
With a more precise measurement, it is easier to find a significant ratio differences
between SIDM models and the CDM model. 
However, in order to compute the numerator of
the Bayes factor, a SIDM simulation with 1) many merging clusters with suitable
configurations, and 2) realistic error propagation is needed.  
Although the analysis a SIDM simulation is out of the scope of this work, 
the analysis can utilize the galaxy summary statistics that we have proposed.  


% Using either the BCG or the KDE peak that I found to be the most
% precise representations of the galaxy populations, it is possible to
% come up with better constraints of $\sigmaSIDM$ based on $\offset_{\rm obs}$. 
% It also requires find merging clusters with favorable configurations 
% in both SIDM simulations and
% observations, which are computationally expensive. It is non-trivial to construct a likelihood
% function to compare the simulated clusters and the observations due to the high
% number of parameters needed to describe a cluster. 
% We have to obtain the joint PDF distribution of the offset
% and the merger configurations before we can do a model comparison. 
% Currently, we do not know the mathematical form and the
% parameters of the PDF of $\offset_{\SIDM}$. There is a need to find this PDF
% from SIDM simulations. If the variance  
% of the $\offset_{\SIDM}$ from the selected cluster
% sample is too big, the variance can be explained many other sources,
% such as the projection uncertainty, the noise from the estimation of
% $\offset$ or the merger phase uncertainty, none of which we have a
% precise quantitative handle on.

\subsection{Modeling cosmic shear using a Gaussian Process}
For a final project, my collaborators and I proposed a new probabilistic method
for inferring a cosmic shear signal (see chapter 4). 
Cosmic shear is the weak 
gravitational lensing signal of background galaxies due to the large-scale DM density field. 
There are many sources of noise and uncertainty that can complicate the observation and
the analysis of the cosmic shear. 
Our proposed method, a GP, is a powerful regression method.
After enforcing the GP covariance kernel to encode the known relations between
the lensing observables, 
we can use the GP simultaneously infer the joint PDF of the signal (both the convergence and the shear)
and the noise.  With a set of simulated and a set of real data, we demonstrated that our GP
method can recover the lensing signals.  
The remaining challenges for extending this method include: 1) reducing the
inherit computational complexity of the method to achieve a speed up. This can
be done via the use of latent variables (inducing points) to achieve a sparse representation of
the lensing observables, and 2) constructing an 
extension to the probabilistic 
framework for fitting cosmological parameters such as $\Omega_m$ and
$\sigma_8$.  With the use of a GP, we can infer and remove the characteristic
patterns (B-mode) caused by 
complicated systematics and noise. This has the potential to give less biased 
cosmological constraints for future surveys that are systematics dominant. 

The field of astrophysics research is always pushing the limit of signal
processing for verifying our scientific theories.
With better statistical models for incorporating our astrophysical knowledge, 
more precise data and increasing computational power, it is promising that the
astrophysics community can make great progress based on my contributions.   


% With the distributions of offsets obtained from my analysis, we were able to perform a
% preliminary hypothesis test to see if uncertainties can explain 
% the level of offset. We showed that 
% the observed level of offsets between the member galaxy population and the
% dark matter (DM) is mostly consistent with Cold Dark Matter cosmology. 

% To make a more scientific statement from the comparison of the model for
% self-interaction dark matter (SIDM) with observations, 
% we need a proper Bayesian model comparison.  
% This requires curating the probability density functions of merger
% configurations with SIDM observables from (cosmological) simulations, 
% where the required merger configurations (\phi), according to Kim et al. include:
% the merging velocities, the concentration of the DM components of the clusters,
% the impact parameter, and the time-since-pericenter.
% We can write down the marginal likelihood term as:
% \begin{equation}
% 	Pr(\Delta s | \sigmaSIDM) = \int Pr(\Delta s | \phi)  Pr(\phi | \sigmaSIDM) d \phi
% \end{equation}
% 
% The proper Bayes factor from model comparison is then: 
% \begin{align}
% 	\frac{Pr(\sigmaSIDM = \sigma | {\bf d})}{Pr(\sigmaSIDM = 0 | {\bf d})} &= 
% 	\frac{Pr()}{}
% \end{align}
% 
% where the evaluation of the evidence term needs to be evaluated separately for
% each $\sigmaSIDM$ model.
% The representation of our knowledge about different physical systems
% affects our ability for verifying our physical theories. 
% Throughout this work, I have shown several instances when the 
% ancillary information from other studies on similar subjects 
% can give us a better understanding of our systems, e.g. the spatial location
% and velocity
% estimates of the radio relic can be used to constrain the merger phase.     
% However, point estimates or qualitative descriptions 
% often give insufficient information for a proper statistical analysis.   
% A next step that I advocate, is to minimize the use of point estimates for
% presenting results, and to present the probability density function (pdf) 
% of the estimated parameters.  how to curate prior information for proper 
% statistical learning 
% to incorporate in a inference framework.

% Bayesian methods can allow a more consistent framework 
% 
% The theoretical computational complexity of an algorithm can be a 
% formidable obstacle. 
% 
% employ dimensionality reduction approach such as inducing points 
% it is possible to gain an understanding of how effects of the cosmological
% parameters are manifested the matter distribution 
% most of the spatial locations of density fluctuations may describe the initial
% condition of the spatial distribution than the cosmological parameters  
