% The following commands produce page numbering at the bottom
% center using roman numerals per UC Davis requirements for the
% front matter of the dissertation:
% \pagenumbering{roman}
% \pagestyle{plain}
% The following command produces double-spaced lines for the
% remainder of the document:
\doublespacing

\setcounter{chapter}{4}
\chapter{Conclusions}{}{}
\label{chapter5}

\epigraph{``Every great and deep difficulty bears in itself its own solution. It forces
us to change our thinking in order to find it."}{Niels Bohr}

\section{Overview}
The main theme of my dissertation is on the analysis and the modeling of
large-scale astrophysical systems. 
Due to the low signal-to-noise of the systems, my work relies on probabilistic
methods for representing both the physical signal and the associated uncertainties. 
% \begin{itemize}
% \item chapter 2: incorporating observation and simulation findings 
% as prior information for tightening parameter estimates 
% \item chapter 3: evaluating the contribution of statistical noise to observations 
% \item chapter 4: signal reconstruction using a Gaussian Process (GP)  
% \end{itemize}

\section{Implications of my projects}
\subsection{Understanding the merger state of galaxy clusters}
We performed a study of the merging galaxy cluster, El Gordo. 
By drawing the distributions of the observed
physical variables as part of a Monte Carlo simulation, 
we evolve the motion of the merger of El Gordo analytically
and gave estimates of the latent variables. 
We (re)discovered how the qualitative spatial location 
of cluster components, such as the cool core 
and the radio relic, can be used to construct   
broad quantitative prior distributions for 
constraining the merger phase.  
We showed that it is not only possible, 
but more likely that the merging galaxy cluster, El Gordo, is in a return phase
than an outgoing phase. 
 
\subsection{Inferring the spatial offsets of the components of galaxy clusters}
A second related but more general analysis of galaxy clusters was done using the
cosmological simulation, the Illustris simulation. 
During the analysis, we found the best 
representation of the galaxy populations of galaxy clusters. 
The Brightest Cluster
Galaxy (BCG) and the peak from a cross-validated kernel density estimate (KDE)
can give high precision estimates of the DM peak to within $\sim$ 30 kpc.
Other methods such as the shrinking aperture peak estimate and the peak estimate 
from the number density have large one-sigma uncertainties $\sim 80$ kpc.   
Knowing the best summary statistic can help pin down the uncertainties from  
the offset between the galaxy population and the dark matter (DM).

The second main finding of the analysis was from the hypothesis test that we
conducted.
It is possible to compute the spatial 
offset between the DM and the galaxies, i.e.
\begin{equation}
	\offset =  {\bf s}_{\rm gal} - {\bf s}_{\rm DM}
\end{equation}
using ${\bf s}_{\rm gal}$ a precise statistic for representing the galaxy population,
and ${\bf s}_{\rm DM}$ the peak of the DM spatial distribution.
We can decompose the observed spatial offsets according to the different
contributions: 
\begin{equation}
	\offset_{\rm obs} = \offset_{\SIDM} + \offset_{\rm dyn} + n + \cdots
\end{equation}
where $\offset_{\SIDM}$ is the offset due to the self-interaction of dark matter, 
$\offset_{\rm dyn}$ is the offset due to dynamical friction, $n$ is the
contribution from noise and measurement uncertainties, while ``$\cdots$" denotes
any other possible contributions that we have not accounted for. 
We have shown that $\offset_{\rm obs} \approx  n > \offset_{\SIDM}$, i.e. 
statistical noise can match observations of merging galaxy clusters, better than 
the self-interaction of DM can match observations. 
The observed offset levels are possible even under  
a Lambda Cold Dark Matter ($\Lambda$CDM) cosmology. 

% With the distributions of offsets obtained from my analysis, we were able to perform a
% preliminary hypothesis test to see if uncertainties can explain 
% the level of offset. We showed that 
% the observed level of offsets between the member galaxy population and the
% dark matter (DM) is mostly consistent with Cold Dark Matter cosmology. 

% To make a more scientific statement from the comparison of the model for
% self-interaction dark matter (SIDM) with observations, 
% we need a proper Bayesian model comparison.  
% This requires curating the probability density functions of merger
% configurations with SIDM observables from (cosmological) simulations, 
% where the required merger configurations (\phi), according to Kim et al. include:
% the merging velocities, the concentration of the DM components of the clusters,
% the impact parameter, and the time-since-pericenter.
% We can write down the marginal likelihood term as:
% \begin{equation}
% 	Pr(\Delta s | \sigmaSIDM) = \int Pr(\Delta s | \phi)  Pr(\phi | \sigmaSIDM) d \phi
% \end{equation}
% 
% The proper Bayes factor from model comparison is then: 
% \begin{align}
% 	\frac{Pr(\sigmaSIDM = \sigma | {\bf d})}{Pr(\sigmaSIDM = 0 | {\bf d})} &= 
% 	\frac{Pr()}{}
% \end{align}
% 
% where the evaluation of the evidence term needs to be evaluated separately for
% each $\sigmaSIDM$ model.

For future studies of galaxy clusters, 
it is desirable to obtain more quantitative version of the 
prior information so a more robust statistical framework can be built for model
comparison. 
I advocate that simulation studies of merging galaxy cluster  
to 1) make the simulation data available for reanalysis, or 2) if there is
enough number of galaxy clusters, provide  the
(joint or conditional) probability density function\footnote{This is {\it not}
the same as a histogram. Histograms are lossy representations of the information 
available. See Chapter 3 for a discussion of why a kernel density estimate or
other regression methods should be used instead.} of the physical quantities
of galaxy clusters. 

\subsection{Cosmic shear inference with a Gaussian Process}
We proposed a new probabilistic method
for inferring a cosmic shear signal in the last chapter (chapter 4) of this dissertation. 
Cosmic shear is the weak 
gravitational lensing signal of background galaxies due to the large-scale DM density field. 
There can be many sources of noise and uncertainty during the observation and
the analysis of the cosmic shear. 
The proposed probabilistic method, a GP, is a powerful regression method that
allows us to simultaneously infer the signal (both the convergence and the shear)
and the noise contribution. 
With a set of simulated and a set of real data, we demonstrated that our GP
method can recover the signal.  
Remaining challenges for extending this method include: 1) reducing the
inherit computational complexity of the method to achieve a speed up. This can
be done via the use of latent variables (inducing points) to achieve a sparse representation of
the physical generating mechanism of the shear field.  
2) constructing the extension to the probabilistic 
framework for fitting cosmological parameters such as $\Omega_m$ and
$\sigma_8$.


% The representation of our knowledge about different physical systems
% affects our ability for verifying our physical theories. 
% Throughout this work, I have shown several instances when the 
% ancillary information from other studies on similar subjects 
% can give us a better understanding of our systems, e.g. the spatial location
% and velocity
% estimates of the radio relic can be used to constrain the merger phase.     
% However, point estimates or qualitative descriptions 
% often give insufficient information for a proper statistical analysis.   
% A next step that I advocate, is to minimize the use of point estimates for
% presenting results, and to present the probability density function (pdf) 
% of the estimated parameters.  how to curate prior information for proper 
% statistical learning 
% to incorporate in a inference framework.

% Bayesian methods can allow a more consistent framework 
% 
% The theoretical computational complexity of an algorithm can be a 
% formidable obstacle. 
% 
% employ dimensionality reduction approach such as inducing points 
% it is possible to gain an understanding of how effects of the cosmological
% parameters are manifested the matter distribution 
% most of the spatial locations of density fluctuations may describe the initial
% condition of the spatial distribution than the cosmological parameters  
